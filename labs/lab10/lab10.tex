\documentclass[12pt]{article}
\usepackage[margin=1in]{geometry}
\usepackage{fancyvrb}
\usepackage{multicol}
\usepackage{hyperref}
\usepackage{amsmath}
\usepackage{amsfonts}

\usepackage[listings]{tcolorbox}

\definecolor{codegreen}{rgb}{0,0.6,0}
\definecolor{codegray}{rgb}{0.5,0.5,0.5}
\definecolor{codepurple}{rgb}{0.58,0,0.82}
\definecolor{backcolour}{rgb}{0.95,0.95,0.92}

\lstdefinestyle{mystyle}{
    language=Python,
    backgroundcolor=\color{backcolour},   
    commentstyle=\color{codegreen},
    keywordstyle=\color{magenta},
    numberstyle=\tiny\color{codegray},
    stringstyle=\color{codepurple},
    basicstyle=\ttfamily\footnotesize,
    breakatwhitespace=false,         
    breaklines=true,                 
    captionpos=b,                    
    keepspaces=true,                 
    numbers=left,                    
    numbersep=5pt,                  
    showspaces=false,                
    showstringspaces=false,
    showtabs=false,                  
    tabsize=2,
    escapechar=|,
    frame=single
}

\lstset{style=mystyle}

\newcommand{\showfig}[2]{
\noindent\includegraphics[width=\textwidth]{#1}
\centerline{#1}
}

\begin{document}
\sloppy
\centerline{\Large CSCI 111, Lab 10}
\centerline{\large Vector Class}


\begin{description}
\item[Due date:] Midnight, Tuesday, November 29, on Canvas.
No late work accepted.  

\item[File names:]  Names of files, functions, and variables, 
when specified,
must be EXACTLY as specified.  This includes simple mistakes such
as capitalization.

\item[Individual work:]  All work must be your own.  Do not share
code with anyone other than the instructor and teaching assistants.
This includes looking over shoulders at screens with the code open.
You may discuss ideas, algorithms, approaches, {\em etc.} with
other students but NEVER actual code.

\item[A 3d vector type:] Write a module called 
\lstinline{vector.py} such that the following
interaction is possible:

\begin{lstlisting}
>>> from vector import V
>>> v1 = V(1, 2, 3)
>>> v1
<vector.V object at 0x00000189527E4808>
>>> print(v1)
V(1, 2, 3)
>>> print(10 * v1)
V(10, 20, 30)
>>> print(v1 / 5)
V(0.2, 0.4, 0.6)
>>> v2 = V(10,20,30)
>>> print(v2)
V(10, 20, 30)
>>> print(v1 + v2)
V(11, 22, 33)
>>> print(v2 - v1)
V(9, 18, 27)
>>> v1 * v1
14
>>> v1 * v2
140
>>> v2 * v2
1400
\end{lstlisting}

And so on.  Clearly you will be defining a new
class and overloading
the operators.  Make sure you overload
all of the following:

\begin{minipage}{0.4\textwidth}
\begin{verbatim}
+     __add__(self, other)
-     __sub__(self, other)
*     __mul__(self, other)
/     __truediv__(self, other)
==    __eq__(self, other)
!=    __ne__(self, other)
\end{verbatim}
\end{minipage}\hfill
\begin{minipage}{0.4\textwidth}
\begin{verbatim}
-=    __isub__(self, other)
+=    __iadd__(self, other)
*=    __imul__(self, other)
/=    __idiv__(self, other)
-     __neg__(self)
+     __pos__(self)
\end{verbatim}
\end{minipage}

Adding and subtracting is done with two vectors.
They are computed componentwise.

Multiplying and dividing is done with a vector
and a number (float or int).

Two vectors are equal
if all their components are equal, and not
equal otherwise.  Because we will have
lots of floats in our vectors, use \lstinline{math.isclose}
to compare components instead of \lstinline{==}.
You can also use your own function to 
determine if two floats are almost equal.

There is also vector multiplication to be supported.
This is the inner product or dot product
of two vectors:
\[
(a_x, a_y, a_z) \cdot (b_x, b_y, b_z) = a_xb_x + a_yb_y + a_zb_z
\]
Clearly the dot product of two vectors is a scalar
(int or float).  Since there is no ``$\cdot$'' available
in python, you will override the multiplication
symbol, ``{\tt +}''.  This means we have three kinds
of multiplication, where {\tt v} is a vector and {\tt x}
is a scalar:
\begin{lstlisting}
x * v                      v * x                       v * v
\end{lstlisting}
One of these can be handled with \lstinline{__rmul__}.
To handle all three you will need to use type-based
dispatch, where you examine the type of one or more
of the arguments and take action accordingly.

\lstinline{__imul__} obviously does not apply
to vector multiplication.  Why not?

The last two operators in the list above
are the unary operators, so we can write,
for example, \lstinline{v + (-w)} as well as \lstinline{v - w}
or \lstinline{v - -w} instead of \lstinline{v+w}.

None of the above operators should be destructive.
They should all be pure functions
that return new objects and not modify
their arguments.

\item[Testing:]  Write a \lstinline{unittest}
module called \lstinline{test_vector.py}
to accompany your \lstinline{vector}
module.  Make sure your unit tests test every operator!

\item[Normalize:]  Write a destructive method
\lstinline{normalize(self)}
that will normalize a vector, {\em i.e.} stretch
or shrink the vector so that its length is 1.
If $v$ is a vector, length of 
the  vector, $|v|$, can be computed by
\[
|v| = \sqrt{v\cdot v}
\]
A vector can be made unit length by dividing by its
length.  If the length of the vector is zero,
you should raise a \lstinline{RunTimeError}
with the message \lstinline{'cannot normalize zero vector'}
rather than the normal divide by zero error.

This method should be destructive, the vector
itself should be modified.  No new vector is returned.


\item[Testing:] 
Add tests to your \lstinline{test_vector} unittest
module to test normalization.  Make sure all
tests are passed before
you continue!

\item[Project:] Write a nondesctructive method
\lstinline{project(self, other)}
that will find the projection of a vector onto another.
Mathematically the projection of $v$  onto $w$
is
\[ \mbox{proj}_w(v) = \frac{v \cdot w}{w \cdot w} w \]

This method is nondestructive, returns a new
vector and does not modify either of the input
vectors.


\item[Testing:] 
Add tests to your \lstinline{test_vector}
module to test projection.   Make sure all
tests are passed before
you continue!

\item[Gram-Schmidt process:]  The Gram-Schmidt
process starts with any three linearly independent
vectors and ends up with three orthonormal vectors.
Orthonormal means they are all of unit length
and all perpendicular to each other.

You can test whether two vectors $v$ and $w$ are perpendicular
by testing whether $v\cdot w = 0$.  Since we're using
floating point numbers a lot with vectors, don't use equality,
\lstinline{==} to compare to zero, use \lstinline{math.isclose},
or your own function to compare two floats to see
if they are almost the same.

If $v_1, v_2, v_3$ are the original vectors, the Gram-Schmidt
process works as follows:

\begin{align*}
u_1 &= v_1\\
u_2 &= v_2 - \mbox{proj}_{u_1}(v_2)\\
u_3 &= v_3 - \mbox{proj}_{u_1}(v_3)- \mbox{proj}_{u_2}(v_3)
\end{align*}

If any one of these vectors $u_1, u_2, u_3$ is (close to) zero,
the original vectors are not (really) linearly indepedent.

After this, the three vectors $u_1, u_2, u_3$ are
normalized to unit length.

Write a method \lstinline{gram_schmidt(self, v2, v3)}
that will desctructively perform the above algorithm
on the three input vectors, ending up with all of them
orthonormal. 

If you find that the original vectors are not
linearly independent, raise the \lstinline{RuntimeError}
with the message 
\lstinline{'cannot orthonormalize linearly dependent vectors'}

\item[Testing:] 
Add tests to your \lstinline{test_vector}
module to test the Gram-Schmidt process.  Make
sure you test orthonomrality and linear 
independence for several sets of vectors. Make sure all
tests are passed before
you turn in the assignment!

Here's some data you can turn into a test:
\begin{lstlisting}
    v1 = V(1,1,1)*3
    v2 = V(1,-2,3)*4
    v3 = V(1,3,-2)*5
    v1.gram_schmidt(v2,v3)
    for v in (v1, v2, v3):
        print(v)
        print(v*v)
\end{lstlisting}
\begin{lstlisting}
V(0.57735, 0.57735, 0.57735)
1.0
V(0.0936586, -0.749269, 0.65561)
1.0
V(0.811107, -0.324443, -0.486664)
1.0000000000000002
\end{lstlisting}

\end{description}
\end{document}
