\documentclass[12pt]{article}
\usepackage[margin=1in]{geometry}
\usepackage{fancyvrb}
\usepackage{multicol}

\usepackage[listings]{tcolorbox}

\definecolor{codegreen}{rgb}{0,0.6,0}
\definecolor{codegray}{rgb}{0.5,0.5,0.5}
\definecolor{codepurple}{rgb}{0.58,0,0.82}
\definecolor{backcolour}{rgb}{0.95,0.95,0.92}

\lstdefinestyle{mystyle}{
    language=Python,
    backgroundcolor=\color{backcolour},   
    commentstyle=\color{codegreen},
    keywordstyle=\color{magenta},
    numberstyle=\tiny\color{codegray},
    stringstyle=\color{codepurple},
    basicstyle=\ttfamily\footnotesize,
    breakatwhitespace=false,         
    breaklines=true,                 
    captionpos=b,                    
    keepspaces=true,                 
    numbers=left,                    
    numbersep=5pt,                  
    showspaces=false,                
    showstringspaces=false,
    showtabs=false,                  
    tabsize=2,
    escapechar=|,
    frame=single
}

\lstset{style=mystyle}
\begin{document}
\sloppy
\centerline{\Large CSCI 111, Lab 2}
\centerline{\large Geoffrey Matthews}

\begin{description}
\item[Due date:] Midnight, Tuesday, September 20, on Canvas.
No late work accepted.  

\item[File names:]  Names of files and variables, when specified,
must be EXACTLY as specified.  This includes simple mistakes such
as capitalization.

\item[Individual work:]  All work must be your own.  Do not share
code with anyone other than the instructor and teaching assistants.
This includes looking over shoulders at screens with the code open.
You may discuss ideas, algorithms, approaches, {\em etc.} with
other students but NEVER actual code.

\item[Setup:] Log in to a computer in the lab and launch
Python's IDLE from the menu.  The interpreter window will have
a title bar of \fbox{\sf IDLE Shell 3.10.0}, where the version
number may be different, but should start with 3.



\item[The editor:]  Running programs in the shell
is good for small tasks and makes a good replacement for
a calculator.  However, if we're going to write long
programs we need to write the program out and save
it in a file.  For this we use the IDLE editor.
\begin{itemize}
\item In the menu bar of the IDLE shell, click on
\fbox{\sf File}$\rightarrow$\fbox{\sf New}.
\item A new window will open up with the title \fbox{\sf Untitled}.
\item Use \fbox{\sf File}$\rightarrow$\fbox{\sf Save} to
save your file.  Unless you did this for lab01,
make a new folder in your home directory,
or your Box directory, called \verb|CSCI111Labs|
\item Make a new folder in \verb|CSCI111Labs| called
\verb|lab02|
\item Finally, save the file as \verb|sphere.py|
\end{itemize}

\item[The problems:] The problems all come from 
exercises for chapter 2 of the text, however 
you are required to solve the problems in a general
way using functions.  You will turn in three
modules: \verb|sphere.py|, \verb|wholesale.py|,
and \verb|home.py|

\item[sphere.py:] In the file saved as \verb|sphere.py|:
\begin{itemize}
\item
The volume of a sphere with radius r is $\frac{4}{3} \pi r^3$.
\item
Use the IDLE shell to find the volume of a sphere with radius 5.
\item
In the \verb|sphere.py| file open in your editor window,
define a {\bf function} \verb|sphere|.
\item When you run your file, it will print no output, but
\item the \verb|sphere| function will be defined
so as to print a message like this:
\begin{lstlisting}
>>> sphere(5)
The volume of a sphere of radius 5 is 523.5987755982989
\end{lstlisting}
\item Check to make sure your function got the same
value you did in the IDLE shell.
\end{itemize}

\item[wholesale.py:] Save a new file with the name \verb|wholesale.py|
\begin{itemize}
\item 
Suppose the cover price of a book is \$24.95, but bookstores get a 40\% discount. Shipping costs \$3 for the first copy and 75 cents for each additional copy.
\item Use the IDLE shell to find out the total wholesale cost for 60 copies.
You should get the following price, in pennies:
\begin{lstlisting}
300+75*59+int(0.4*2495)*60
\end{lstlisting}
\item Note that we have assumed the discount discards
any fractions of a cent.
\item Now write three functions with the following headers:
\begin{lstlisting}
def shipping_cost(n, first_copy, extra_copy):
\end{lstlisting}
\begin{lstlisting}
def wholesale_cost(n, cover_price, discount):
\end{lstlisting}
\begin{lstlisting}
def total_cost(n, cover_price, discount, first_copy, extra_copy):
\end{lstlisting}
which do the obvious things, where the total cost
is just the sum of the wholesale cost and the shipping cost.
\item The original problem can now be solved by running the
module and then entering in the shell:
\begin{lstlisting}
print(total_cost(60, 2495, 40, 300, 75))
\end{lstlisting}
You should get the same answer as before.
\item Note that the prices are all in pennies and the
percentage is a percentage, not a fraction.
\item Here are some results from my solution to check your work:
\begin{lstlisting}
>>> total_cost(100, 2495, 40, 300, 75)
107525
>>> total_cost(20, 32595, 25, 100, 50)
164010
>>> total_cost(1000, 195, 10, 100, 5)
24095
\end{lstlisting}
\end{itemize}

\item[home.py:]
If I leave my house at 6:52 am and run 1 mile at an easy pace (8:15 per mile), then 3 miles at tempo (7:12 per mile) and 1 mile at easy pace again, what time do I get home for breakfast?

\begin{itemize}

\item Solve the above problem using the shell.
\item Open a file called \verb|home.py|
and write {\bf several appropriate functions} to help
 solve the above problem.
\item You can make simplifying assumptions,
for example that there are only ever two
paces, but that each one might be different and for
different distances.
\item Your computations should be in minutes as integers.
\item Document each function with comments.
\item Include a \verb|print| statement at the end of the file
that solves the above problem, and then several others.
\end{itemize}

\item[Requirements:]
\begin{itemize}
\item
Each of the three files, \verb|sphere.py|,
\verb|wholesale.py|, and \verb|home.py|
should begin with a comment block with
the information like this:
\begin{lstlisting}
#  home.py
#  Geoffrey Matthews
#  CSCI 111, Fall 2022
#  functions to calculate arrival at home
#  after running at varying paces
\end{lstlisting}
\begin{center}
\fbox{\fbox{
\begin{minipage}{0.5\textwidth}\bf
All programs you hand in should begin with
a similar comment block!
\end{minipage}}}
\end{center}

\end{itemize}

\item[Upon completion:] Zip your \verb|lab02| folder
with the three programs, \verb|sphere.py|,
\verb|wholesale.py|, and \verb|home.py|
and turn in the zipped file to Canvas.

\item[Zipping:] You can zip a folder in windows by right-clicking
on the folder in the file browser and selecting from the popup menu:

\fbox{\sf Send to}\ensuremath{\rightarrow}\fbox{\sf Compressed (zipped) folder}


\end{description}
\end{document}

