\documentclass[12pt]{article}
\usepackage[margin=1in]{geometry}
\usepackage{fancyvrb}
\usepackage{multicol}
\usepackage{hyperref}

\usepackage[listings]{tcolorbox}

\definecolor{codegreen}{rgb}{0,0.6,0}
\definecolor{codegray}{rgb}{0.5,0.5,0.5}
\definecolor{codepurple}{rgb}{0.58,0,0.82}
\definecolor{backcolour}{rgb}{0.95,0.95,0.92}

\lstdefinestyle{mystyle}{
    language=Python,
    backgroundcolor=\color{backcolour},   
    commentstyle=\color{codegreen},
    keywordstyle=\color{magenta},
    numberstyle=\tiny\color{codegray},
    stringstyle=\color{codepurple},
    basicstyle=\ttfamily\footnotesize,
    breakatwhitespace=false,         
    breaklines=true,                 
    captionpos=b,                    
    keepspaces=true,                 
    numbers=left,                    
    numbersep=5pt,                  
    showspaces=false,                
    showstringspaces=false,
    showtabs=false,                  
    tabsize=2,
    escapechar=|,
    frame=single
}

\lstset{style=mystyle}
\begin{document}
\sloppy
\centerline{\Large CSCI 111, Lab 6}
\centerline{\large Connect Four!}

\begin{description}
\item[Due date:] Midnight, Tuesday, October 25, on Canvas.
No late work accepted.  

\item[File names:]  Names of files, functions, and variables, 
when specified,
must be EXACTLY as specified.  This includes simple mistakes such
as capitalization.

\item[Individual work:]  All work must be your own.  Do not share
code with anyone other than the instructor and teaching assistants.
This includes looking over shoulders at screens with the code open.
You may discuss ideas, algorithms, approaches, {\em etc.} with
other students but NEVER actual code.

\item[Connect Four:] I'm going to assume no one needs an explanation
of the game.  It is a simple variation on tic-tac-toe played on a 6 row by 
7 column board, you need to get four in a row to win, and each play
occupies the lowest unoccupied square in its column.

\item[Turn in:] Write a single module, \lstinline{connect4.py},
for this assignment.  Zip it in a folder titled \lstinline{lab06}.

\item[GUI Implementation:]  You will make this with a graphical
interface in Tkinter.  Everything you need to get started
is provided with the \lstinline{tictactoe.py} module 
in this lab's folder.  You should follow my strategy exactly for
representing places on the board with buttons, states with dictionaries,
{\em etc.}, but
generalize and modify it as follows.

\item[Board size:]  Before the game starts, ask the user for
the number of rows and columns.  Default values should be 6 rows
and 7 columns.  You can use two \lstinline{simpledialog}'s for this,
as in my tic tac toe game, or make a fancier dialog with tkinter.

\item[Gravity:] The user can click on any unoccupied square, but
the mark will appear in the lowest empty square.

You do not have to animate the falling mark!  That's for graphics
or games classes.

\item[Check for winner:] In my tic tac toe game I use a brute force
algorithm to see if there's a winner.  I check all possible winning
moves to see if one player has won.  With connect four, on a reasonable
size board, this would be really slow.  (How many possible wins would
you have to check?)  Instead, if we assume that we check for a win
after each move, we need check only four possible wins:
\begin{itemize}
\item After a move, the only possible win is by the player that
just moved.
\item After a move, the only possible win must involve the
position just moved.
\item There are only four possible directions a win can go: 
vertical, horizontal, northwest-southeast, and southwest-northeast.
\end{itemize}

\item[Check winner algorithm:] Follow the following procedure.
You will have to solve several programming problems to translate
this informal description into code.

\begin{enumerate}
\item For each of the possible winning
directions, start in the topmost, leftmost, northwestmost, or southwestmost
position on the board that is aligned with the last move position.  

Do {\em not} try to start just 3 squares away from the last move.
Start at the very edge of the board.  You will have to figure
out how to find this square.

\item Step along the winning direction one square at a time.
If you started in the right spot, you should hit the square
for the most recent move along the way.
\item When you hit any square with the same color of the last move,
start counting.
\item Keep counting until you hit a square not the last mover's color.
\item If you counted to 4 or beyond, winner!
\item Otherwise, keep stepping along the direction looking for
streaks of 4 or more until you're off the other side of the 
board.
\item Checking for a cat's game (no winner) is really easy
if you keep track of the number of moves so far.
\end{enumerate}

You may recognize the search pattern involved here from previous labs!

\item[Notes:]~

\begin{itemize}
\item  You will probably need a reverse dictionary (mapping
buttons to positions), not implemented in my tic tac toe game.
\item You may want to make the number of rows and columns 
global variables.
\end{itemize}



\end{description}
\end{document}
