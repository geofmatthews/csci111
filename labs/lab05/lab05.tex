\documentclass[12pt]{article}
\usepackage[margin=1in]{geometry}
\usepackage{fancyvrb}
\usepackage{multicol}
\usepackage{hyperref}

\usepackage[listings]{tcolorbox}

\definecolor{codegreen}{rgb}{0,0.6,0}
\definecolor{codegray}{rgb}{0.5,0.5,0.5}
\definecolor{codepurple}{rgb}{0.58,0,0.82}
\definecolor{backcolour}{rgb}{0.95,0.95,0.92}

\lstdefinestyle{mystyle}{
    language=Python,
    backgroundcolor=\color{backcolour},   
    commentstyle=\color{codegreen},
    keywordstyle=\color{magenta},
    numberstyle=\tiny\color{codegray},
    stringstyle=\color{codepurple},
    basicstyle=\ttfamily\footnotesize,
    breakatwhitespace=false,         
    breaklines=true,                 
    captionpos=b,                    
    keepspaces=true,                 
    numbers=left,                    
    numbersep=5pt,                  
    showspaces=false,                
    showstringspaces=false,
    showtabs=false,                  
    tabsize=2,
    escapechar=|,
    frame=single
}

\lstset{style=mystyle}
\begin{document}
\sloppy
\centerline{\Large CSCI 111, Lab 5}
\centerline{\large RPN Calculator}

\begin{description}
\item[Due date:] Midnight, Tuesday, October 11, on Canvas.
No late work accepted.  

\item[File names:]  Names of files, functions, and variables, 
when specified,
must be EXACTLY as specified.  This includes simple mistakes such
as capitalization.

\item[Individual work:]  All work must be your own.  Do not share
code with anyone other than the instructor and teaching assistants.
This includes looking over shoulders at screens with the code open.
You may discuss ideas, algorithms, approaches, {\em etc.} with
other students but NEVER actual code.

\item[RPN:] Reverse Polish Notation is a way of expressing
algebraic expressions without using parentheses.
Instead of coming between operands, the operator comes
after the operands.

\begin{tabular}{rcl}
4 + 5 & becomes & 4 5 + \\
(4 + 5) * 3 & becomes & 4 5 + 3 * \\
4 + (5 * 3) & becomes & 4 5 3 * + \\
sin(123)**2 + cos(123)**2 & becomes & 123 cos 2 ** 123 sin 2 ** + \\
\end{tabular}

There are a lot of examples online.  You should make sure you understand
how it works before attempting this lab.

\item[RPN calculators:] Because no parentheses are required,
the interface to a calculator was much simpler.  HP calculators
in the early years, and some even today, used RPN.

Start a module \lstinline{rpn.py} which will need to import
\lstinline{math}:
\begin{lstlisting}
import math
\end{lstlisting}  
We will put the following functions into this module.

\item[Parse numbers:] Before we can process a line of RPN,
we need to parse it.  This involves breaking it into
tokens (easy, tokens are separated by whitespace, so use
{\tt split}), and then convert the numeric strings
into actual numbers.   For instance, the string \lstinline{'2.3e2'}
has to be translated into the number \lstinline{230.0}.

\item[Parse integers:]
We break this down into several simple parts.  Integers
are easy: they consist of an optional \lstinline{'+'} or
\lstinline{'-'}, followed by any number of digits
in \lstinline{'0123456789'}.  Each digit represents a quantity ten
times bigger than the one to its right, so we can
process an integer  as follows:
\begin{enumerate}
\item Set variable \lstinline{sign} to \lstinline{1}
\item If the first character is \lstinline{'+'}, remove it.
\item If the first character is \lstinline{'-'}, remove it and 
set \lstinline{sign} to \lstinline{-1}
\item Set variable \lstinline{accumulator} to \lstinline{0}
\item For each of the remaining digits, \lstinline{d}:
\begin{enumerate}
\item Multiply \lstinline{accumulator} by \lstinline{10}
\item Add \lstinline{ord(d) - ord('0')} to \lstinline{accumulator}
\end{enumerate}
\item Return \lstinline{sign * accumulator}
\end{enumerate}
Note that the empty string should parse as the integer \lstinline{0}.
This will come in handy.

Write a function \lstinline{parse_integer} that does this:
\begin{lstlisting}
>>> parse_integer('321')
321
>>> parse_integer('-555')
-555
>>> parse_integer('')
0
\end{lstlisting}


\item[Parse floats:] Floats are more difficult because
they have more parts.  A typical float, \lstinline{+123.456e-78}
for example, looks like:
\[
\underbrace{+123}_{whole}%
\underbrace{.}_{dot}\underbrace{456}_{fraction}%
\underbrace{e}_{e}\underbrace{-78}_{exponent}
\]
So, really it consists of three integers, \lstinline{whole},
\lstinline{fraction}, and \lstinline{exponent}.
The \lstinline{e} and the \lstinline{exponent} are optional.

In python either the \lstinline{whole} or the \lstinline{fraction}
is optional, but you can't omit both!  We will let both of
them be optional for simplicity's sake.

So, now we have a simple algorithm for integers:
\begin{enumerate}
\item Set the variable \lstinline{exponent} to \lstinline{0}
\item If \lstinline{'e'} is in the string:
\begin{enumerate}
\item Split on \lstinline{'e'}
\item Parse the second part as an integer
\item Set the variable \lstinline{exponent} to this integer
\item Set the string to the first part
\end{enumerate}
\item If \lstinline{'.'} is in string:
\begin{enumerate}
\item Split the string on \lstinline{'.'}
\item Parse the first part as an integer
\item Assign that to \lstinline{whole}
\item Assign the length of the second part to \lstinline{n}
\item Parse the second part as an integer
\item Assign that to \lstinline{fraction}
\item Divide \lstinline{fraction} by $10^{\mbox{\tt n}}$
\item If \lstinline{whole} is negative, multiply \lstinline{fraction} by
\lstinline{-1}.
\end{enumerate}
\item If \lstinline{'.'} is not in the string:
\begin{enumerate}
\item Parse the string as an integer
\item Set \lstinline{whole} to this integer
\item Set \lstinline{fraction} to \lstinline{0}
\end{enumerate}
\item Return \lstinline{(whole + fraction) * (10 ** exponent)}
\end{enumerate}

Write a function \lstinline{parse_float} that does this:
\begin{lstlisting}
>>> parse_float('1e1')
10.0
>>> parse_float('-123e-123')
-1.2300000000000001e-121
>>> parse_float('345.678e4')
3456780.0
>>> parse_float('-33.44e-2')
-0.3344
>>> parse_float('.33e2')
33.0
>>> parse_float('.')
0.0
\end{lstlisting}

\item[Parsing numbers:] A string in our calculator could be
either an integer or a float.  Write a function \lstinline{parse_number}
that will take a string of either kind and return either
an integer or a float.  Examples:
\begin{lstlisting}
>>> parse_number('1e1')
10.0
>>> parse_number('10')
10
\end{lstlisting}
We will assume that all input is either a valid integer or a
valid float.  How can you tell them apart?  Hint:
floats must have either \lstinline{'.'} or \lstinline{'e'}
in them.

Note: our numbers are slightly different from Python's numbers.
For example, what does Python do with \lstinline{'.'}?
How about \lstinline{'0003'}?  Our parser will have different
answers.

Note:  once again, we will assume the input is perfect.
We will not check for errors.  So if we try to parse
a number out of the string \lstinline{'1.0xxx99'}
the program will crash.  (What will be the error?)

\item[Keywords:] There will be certain strings that
are keywords in our RPN calculator language.  They are:
\begin{lstlisting}
['pop', 'clr', '+', '-', '*', '/', '**', 'sin', 'cos', 'sqrt']
\end{lstlisting}

Write a boolean function \lstinline{is_keyword}  that determines
whether or not a string is a keyword.

\item[Tokenize:] We need to take a line of input and
convert it to tokens.  The tokens of our language are
integers, floats, and keywords.  The first step is easy, since
these are all separated by whitespace in our RPN calculator.
Keywords are represented
by strings, but the strings for numbers should be converted
to numbers of the right type.
Write a function \lstinline{tokenize} that takes a string
and returns a list of tokens.  For example:
\begin{lstlisting}
>>> tokenize('3 99.9 + 1e5 *')
[3, 99.9, '+', 100000.0, '*']
\end{lstlisting}

Note: We will assume that all input is perfect.  You
don't have to check for things that are not 
keywords or numbers.

\item[The stack:]  Our interpreter will use a global variable
called \lstinline{The_Stack}.  This will behave like a {\bf stack}
data structure,
a particular kind of list where you can {\em only}
remove one item from the end, or instert one item
on the end.  In other words, the {\bf only} operations
allowed on a stack \lstinline{s} are:
\lstinline{s.pop()} and \lstinline{s.append(item)}.
Other than initializing it to an empty list, and printing it,
you are not allowed to do anythgin else to a stack.

\item[The calculator:]  An RPN calculator works as follows:
\begin{enumerate}
\item Read a token.  
\item If the token is a number, push it onto the stack.
\item If the token is an operator:
  \begin{enumerate}
  \item If it is \lstinline{'+'}:
    \begin{enumerate}
    \item Pop two items from the stack
    \item Add them up
    \item Push the sum onto the stack
    \end{enumerate}
  \item If it is \lstinline{'-'} or \lstinline{'*'} or ...,
    \begin{enumerate}
    \item behave similarly to \lstinline{'+'} (be careful about
    argument order for \lstinline{'-'} {\em etc.})
   \end{enumerate}
  \item If it is \lstinline{'sin'} or \lstinline{'sqrt'} or ...
    \begin{enumerate}
    \item behave similarly to \lstinline{'+'} but with only one item from the stack.
    \end{enumerate}
  \item If it is \lstinline{'pop'}:
    \begin{enumerate}
    \item Pop the tack
   \end{enumerate}
  \item If it is \lstinline{'clr'}:
    \begin{enumerate}
    \item Remove all items from the stack
   \end{enumerate}
  \end{enumerate}
\item Loop to step 1
\end{enumerate}

Write a function \lstinline{rpnline} that will take a line of
text as a string and a stack as arguments.
It then tokenizes the string,  and proceeds to handle
all the tokens in the manner above.

Before handling each token, have the procedure print
the stack and the remaining tokens.  It really helps understanding
and debugging.

Once again, we are not concerned with errors. If someone
enters the string \lstinline{'99 +'} starting with an empty stack,
then the program will crash.  (What would be the error?)

When out of tokens, the function prints the stack one more time
before terminating.

Example:
\begin{lstlisting}
>>> The_Stack = []
>>> rpnline('3 4 1e1 + *', The_Stack)
Stack: [] Tokens: [3, 4, 10.0, '+', '*']
Stack: [3] Tokens: [4, 10.0, '+', '*']
Stack: [3, 4] Tokens: [10.0, '+', '*']
Stack: [3, 4, 10.0] Tokens: ['+', '*']
Stack: [3, 14.0] Tokens: ['*']
[42.0]
\end{lstlisting}

\item[REPL:]  You can now build your
own REPL, like IDLE's, but for your RPN calculator:
\begin{lstlisting}
def repl(stack):
    while True:
        line = input('RPN>>> ')
        if line == 'exit':
            break
        rpnline(line,stack)
\end{lstlisting}
You can run it like this:
\begin{lstlisting}
>>> repl(list())
RPN>>> 4 5 *
Stack: [] Tokens: [4, 5, '*']
Stack: [4] Tokens: [5, '*']
Stack: [4, 5] Tokens: ['*']
[20]
RPN>>> 2 2 2 2 ** ** **
Stack: [20] Tokens: [2, 2, 2, 2, '**', '**', '**']
Stack: [20, 2] Tokens: [2, 2, 2, '**', '**', '**']
Stack: [20, 2, 2] Tokens: [2, 2, '**', '**', '**']
Stack: [20, 2, 2, 2] Tokens: [2, '**', '**', '**']
Stack: [20, 2, 2, 2, 2] Tokens: ['**', '**', '**']
Stack: [20, 2, 2, 4] Tokens: ['**', '**']
Stack: [20, 2, 16] Tokens: ['**']
[20, 65536]
RPN>>> exit
\end{lstlisting}
Notice I didn't  clear the stack between operations.
That's OK; you might want to make several computations,
leave all the results on the stack, and then
do further computations with the results.

\item[File processing:]  You can also put 
RPN programs in a file and run them all like
this:
\begin{lstlisting}   
def runfile(fname,stack):
    fin = open(fname)
    for line in fin:
        line = line.strip()
        if line == 'exit':
            break
        print('INPUT:', line)
        rpnline(line,stack)
\end{lstlisting}

\item[Turn in:] Turn in the single module \lstinline{rpn.py}.
I will run it on a new input file during grading,
so be sure yours can handle all proper input, 
using any of the operators in the list of keywords.

\item[Sample output:] Here is the output from my program
on a sample input file, similar to the one I will use for
grading:
\end{description}
\begin{lstlisting}
INPUT: 009 9 *
Stack: [] Tokens: [9, 9, '*']
Stack: [9] Tokens: [9, '*']
Stack: [9, 9] Tokens: ['*']
[81]
INPUT: clr
Stack: [81] Tokens: ['clr']
[]
INPUT: 5 4 3 - -
Stack: [] Tokens: [5, 4, 3, '-', '-']
Stack: [5] Tokens: [4, 3, '-', '-']
Stack: [5, 4] Tokens: [3, '-', '-']
Stack: [5, 4, 3] Tokens: ['-', '-']
Stack: [5, 1] Tokens: ['-']
[4]
INPUT: clr
Stack: [4] Tokens: ['clr']
[]
INPUT: 5 4 - 3 -
Stack: [] Tokens: [5, 4, '-', 3, '-']
Stack: [5] Tokens: [4, '-', 3, '-']
Stack: [5, 4] Tokens: ['-', 3, '-']
Stack: [1] Tokens: [3, '-']
Stack: [1, 3] Tokens: ['-']
[-2]
INPUT: clr
Stack: [-2] Tokens: ['clr']
[]
INPUT: 3 2 ** 4 2 ** + sqrt
Stack: [] Tokens: [3, 2, '**', 4, 2, '**', '+', 'sqrt']
Stack: [3] Tokens: [2, '**', 4, 2, '**', '+', 'sqrt']
Stack: [3, 2] Tokens: ['**', 4, 2, '**', '+', 'sqrt']
Stack: [9] Tokens: [4, 2, '**', '+', 'sqrt']
Stack: [9, 4] Tokens: [2, '**', '+', 'sqrt']
Stack: [9, 4, 2] Tokens: ['**', '+', 'sqrt']
Stack: [9, 16] Tokens: ['+', 'sqrt']
Stack: [25] Tokens: ['sqrt']
[5.0]
INPUT: clr
Stack: [5.0] Tokens: ['clr']
[]
INPUT: 123 sin 2 ** 123 cos 2 ** +
Stack: [] Tokens: [123, 'sin', 2, '**', 123, 'cos', 2, '**', '+']
Stack: [123] Tokens: ['sin', 2, '**', 123, 'cos', 2, '**', '+']
Stack: [-0.45990349068959124] Tokens: [2, '**', 123, 'cos', 2, '**', '+']
Stack: [-0.45990349068959124, 2] Tokens: ['**', 123, 'cos', 2, '**', '+']
Stack: [0.21151122074847095] Tokens: [123, 'cos', 2, '**', '+']
Stack: [0.21151122074847095, 123] Tokens: ['cos', 2, '**', '+']
Stack: [0.21151122074847095, -0.8879689066918555] Tokens: [2, '**', '+']
Stack: [0.21151122074847095, -0.8879689066918555, 2] Tokens: ['**', '+']
Stack: [0.21151122074847095, 0.7884887792515292] Tokens: ['+']
[1.0]
INPUT: clr
Stack: [1.0] Tokens: ['clr']
[]
INPUT: 3.14159265 2 / sin
Stack: [] Tokens: [3.1415926499999998, 2, '/', 'sin']
Stack: [3.1415926499999998] Tokens: [2, '/', 'sin']
Stack: [3.1415926499999998, 2] Tokens: ['/', 'sin']
Stack: [1.5707963249999999] Tokens: ['sin']
[1.0]
INPUT: clr
Stack: [1.0] Tokens: ['clr']
[]
INPUT: 3.14159265 2 / cos
Stack: [] Tokens: [3.1415926499999998, 2, '/', 'cos']
Stack: [3.1415926499999998] Tokens: [2, '/', 'cos']
Stack: [3.1415926499999998, 2] Tokens: ['/', 'cos']
Stack: [1.5707963249999999] Tokens: ['cos']
[1.7948967369654108e-09]
INPUT: clr
Stack: [1.7948967369654108e-09] Tokens: ['clr']
[]
INPUT: 1e10 1e-10 *
Stack: [] Tokens: [10000000000.0, 1e-10, '*']
Stack: [10000000000.0] Tokens: [1e-10, '*']
Stack: [10000000000.0, 1e-10] Tokens: ['*']
[1.0]
INPUT: clr
Stack: [1.0] Tokens: ['clr']
[]
INPUT: -2.2e-3 2.2e-3 +
Stack: [] Tokens: [-0.0022, 0.0022, '+']
Stack: [-0.0022] Tokens: [0.0022, '+']
Stack: [-0.0022, 0.0022] Tokens: ['+']
[0.0]
\end{lstlisting}
\end{document}
