\documentclass{beamer}
\usetheme{Singapore}

%\usepackage{pstricks,pst-node,pst-tree}
\usepackage{amssymb,latexsym}
\usepackage{tikz}
\usepackage{graphicx}
\usepackage{fancyvrb}
\usepackage{hyperref}
\usepackage{fancybox}
\usepackage[listings]{tcolorbox}

\definecolor{codegreen}{rgb}{0,0.6,0}
\definecolor{codegray}{rgb}{0.5,0.5,0.5}
\definecolor{codepurple}{rgb}{0.58,0,0.82}
\definecolor{backcolour}{rgb}{0.95,0.95,0.92}

\lstdefinestyle{mystyle}{
    language=Python,
    backgroundcolor=\color{backcolour},   
    commentstyle=\color{codegreen},
    keywordstyle=\color{magenta},
    numberstyle=\tiny\color{codegray},
    stringstyle=\color{codepurple},
    basicstyle=\ttfamily\footnotesize,
    breakatwhitespace=false,         
    breaklines=true,                 
    captionpos=b,                    
    keepspaces=true,                 
    numbers=left,                    
    numbersep=5pt,                  
    showspaces=false,                
    showstringspaces=false,
    showtabs=false,                  
    tabsize=2,
    escapechar=|,
    frame=single
}

\lstset{style=mystyle}


\newcommand{\bi}{\begin{itemize}}
\newcommand{\li}{\item}
\newcommand{\ei}{\end{itemize}}
\newcommand{\Show}[1]{
\begin{center}
\shadowbox{\begin{minipage}{0.8\textwidth}
          #1
          \end{minipage}}
\end{center}
}
\newcommand{\arrow}{\ensuremath{\rightarrow}}


\newcommand{\img}[2]{\centerline{\includegraphics[width=#1\textwidth]{#2}}}

\newcommand{\bfr}[1]{\begin{frame}[fragile]\frametitle{{ #1 }}}
\newcommand{\efr}{\end{frame}}

\newcommand{\cola}{\begin{columns}\begin{column}{0.5\textwidth}}
\newcommand{\colb}{\end{column}\begin{column}{0.5\textwidth}}
\newcommand{\colc}{\end{column}\end{columns}}


\title{Think Python 2e, Chapter 3 Notes}
\author{Geoffrey Matthews}

\begin{document}

\begin{frame}
\maketitle

\end{frame}

\bfr{Function calls}
\bi
\li Functions take an argument and return a result.
\begin{lstlisting}
>>> type(42)
<class 'int'>
\end{lstlisting}
\li The function is \verb|type|
\li The argument is \verb|42|
\li the return value is \verb|<class 'int'>|
\ei


\end{frame}

\bfr{Functions for changing type}
\begin{lstlisting}>>> int('32')
32
>>> int('Hello')
ValueError: invalid literal for int(): Hello
>>> int(3.99999)
3
>>> int(-2.3)
-2
>>> str(32)
'32'
>>> str(3.14159)
'3.14159'
\end{lstlisting}

\end{frame}

\bfr{Math functions}
\bi
\li Most math functions are in the math {\bf module}.
\li You can only use them if you import the module:
\begin{lstlisting}
>>> import math
>>> dir(math)['__doc__', '__loader__', '__name__', '__package__', '__spec__', 'acos', 'acosh', 'asin', 'asinh', 'atan', 'atan2', 'atanh', 'ceil', 'comb', 'copysign', 'cos', 'cosh', 'degrees', 'dist', 'e', 'erf', 'erfc', 'exp', 'expm1', 'fabs', 'factorial', 'floor', 'fmod', 'frexp', 'fsum', 'gamma', 'gcd', 'hypot', 'inf', 'isclose', 'isfinite', 'isinf', 'isnan', 'isqrt', 'lcm', 'ldexp', 'lgamma', 'log', 'log10', 'log1p', 'log2', 'modf', 'nan', 'nextafter', 'perm', 'pi', 'pow', 'prod', 'radians', 'remainder', 'sin', 'sinh', 'sqrt', 'tan', 'tanh', 'tau', 'trunc', 'ulp']

\end{lstlisting}
\ei
\end{frame}

\bfr{Help on buil-in functions}
\begin{lstlisting}

>>> help(math.sin)
Help on built-in function sin in module math:

sin(x, /)
    Return the sine of x (measured in radians).

\end{lstlisting}


\end{frame}
\bfr{Beware of floating point numbers!}

\bi
\li What do you think this returns?
\begin{lstlisting}
>>> math.sin(math.pi)
\end{lstlisting}
\pause
\begin{lstlisting}
1.2246467991473532e-16
\end{lstlisting}
\pause
\li How about this?
\begin{lstlisting}
>>> math.exp(math.log(22))
\end{lstlisting}
\pause
\begin{lstlisting}
22.000000000000004
\end{lstlisting}
\pause
\li How about this?
\begin{lstlisting}
>>> math.cos(math.pi)
\end{lstlisting}
\pause
\begin{lstlisting}
-1.0
\end{lstlisting}
\ei

\bigskip


\Show{Floating point numbers are almost never exact!}


\end{frame}

\bfr{Defining new functions}
\begin{lstlisting}
def print_lyrics():
    print("I'm a lumberjack, and I'm okay.")
    print("I sleep all night and I work all day.")
\end{lstlisting}

\bi
\li The name of the function is \verb|print_lyrics|
\li The rules for variable and function names are the same.
\li The empty parentheses indicate that this function takes no
arguments.
\li The first line is called the {\bf header}
\li The rest is called the {\bf body}
\li Double quotes let us use single quotes as apostrophes.
\li Normally entered in script mode.
\ei

\end{frame}

\bfr{Functions can call other functions}

\bi
\li Create a module (write a file) with the following
two function definitions.
\begin{lstlisting}
def print_lyrics():
    print("I'm a lumberjack, and I'm okay.")
    print("I sleep all night and I work all day.")    
def repeat_lyrics():
    print_lyrics()
    print_lyrics()
\end{lstlisting}
\li Now run the module.
\li Nothing happens, because you haven't printed anything.
\li Defining a function does not run any code.
\li You have to call the functions to print anything.
\ei
\end{frame}


\bfr{Calling the functions}
\bi
\li In the shell, you can call the functions:
\begin{lstlisting}
>>> print_lyrics()
I'm a lumberjack, and I'm okay.
I sleep all night and I work all day.

>>> repeat_lyrics()
I'm a lumberjack, and I'm okay.
I sleep all night and I work all day.
I'm a lumberjack, and I'm okay.
I sleep all night and I work all day.
\end{lstlisting}

\ei
\end{frame}

\bfr{Defining and calling in the script}
\bi
\li You define the following module, and run it.  What happens?
\begin{lstlisting}
def print_lyrics():
    print("I'm a lumberjack, and I'm okay.")
    print("I sleep all night and I work all day.")    
def repeat_lyrics():
    print_lyrics()
    print_lyrics()
    
repeat_lyrics()
\end{lstlisting}

\ei
\end{frame}

\bfr{Flow of control and stack frames}
\small
\cola
\begin{lstlisting}
def f():
    print(1)
    g()
    i()
    print(2)
def g():
    print(3)
    h()
    i()
    h()
    print(4)
def h():
    print(5)
    i()
    print(6)
def i():
    print(7)    
f()
\end{lstlisting}

\colb
\pause

\begin{minipage}{\textwidth}
\tt
\_\_main\_\_  \\\pause
\_\_main\_\_ \arrow f\\\pause
\_\_main\_\_ \arrow f   \arrow g\\\pause
\_\_main\_\_ \arrow f   \arrow g \arrow  h\\\pause
\_\_main\_\_ \arrow f   \arrow g \arrow  h \arrow i\\\pause
\_\_main\_\_ \arrow f   \arrow g \arrow  h \\\pause
\_\_main\_\_ \arrow f   \arrow g \\\pause
\_\_main\_\_ \arrow f   \arrow g \arrow i\\\pause
\_\_main\_\_ \arrow f   \arrow g \\\pause
\_\_main\_\_ \arrow f   \arrow g \arrow  h\\\pause
\_\_main\_\_ \arrow f   \arrow g \arrow  h \arrow i\\\pause
\_\_main\_\_ \arrow f   \arrow g \arrow  h \\\pause
\_\_main\_\_ \arrow f   \arrow g  \\\pause
\_\_main\_\_ \arrow f   \\\pause
\_\_main\_\_ \arrow f   \arrow i\\\pause
\_\_main\_\_ \arrow f  \\\pause
\_\_main\_\_  \\
\end{minipage}
\colc
\end{frame}

\bfr{Parameters and arguments}
\begin{lstlisting}
def print_twice(bruce):
    print(bruce)
    print(bruce)
\end{lstlisting}
\begin{lstlisting}
>>> print_twice('Spam')
Spam
Spam
>>> print_twice(42)
42
42
>>> print_twice(math.pi)
3.14159265359
3.14159265359
>>> print_twice('Spam '*4)
Spam Spam Spam Spam
Spam Spam Spam Spam
>>> print_twice(math.cos(math.pi))
-1.0
-1.0
\end{lstlisting}
\end{frame}


\bfr{Functions can be passed to functions}
\begin{lstlisting}
def foo():
    print('foo')
    
def do_twice(f):
    f()
    f()
    
do_twice(foo)
\end{lstlisting}
\end{frame}

\bfr{Functions can be passed to functions}
\begin{lstlisting}
def print_fancy(x):
    print('==> ' + x + ' <==')
    
def do_twice(f, x):
    f(x)
    f(x)
    
do_twice(print_fancy, 'Geoffrey the Magnificent')
\end{lstlisting}
\end{frame}


\bfr{Parameters and variables in a function are local}
\cola

\begin{tikzpicture}
\node[anchor=south west, inner sep=0] at (0,0) {
\begin{lstlisting}
x = 11

def foo(y):
    z = 2*y
    print(x,y,z)

foo(2*x)
print(x,y,z)
\end{lstlisting}};
%\draw (0,0) grid (5,5);
\draw (0,2.5) -- (4,2.5) -- (4,1.2) -- (0,1.2) -- cycle;
\end{tikzpicture}
\pause
\colb


{\tt
\begin{tabular}{|r|rcl|}\hline
\_\_main\_\_ & x & \arrow & 11 \\\hline\hline
foo & y & \arrow & 22 \\
    & z & \arrow & 33 \\\hline
    \end{tabular}
    }
    \bigskip
    
    Stack frames
    \colc
    
\Show{\begin{itemize}
\item If you're in the box you can see it!
\item If you're not in the box you can't see it!
\end{itemize}
}

\end{frame}

\bfr{Parameters and variables in a function are local}
\cola

\begin{tikzpicture}
\node[anchor=south west, inner sep=0] at (0,0) {
\begin{lstlisting}
x = 11
def foo(y):
    z = 2*y
    print(x,y,z)
    bar(2*z)
def bar(x):
    y = 2*x
    print(x,y,z)
foo(x)
print(x,y,z)
\end{lstlisting}};
%\draw (0,0) grid (5,5);
\draw (0,3.6) -- (4,3.6) -- (4,2.1) -- (0,2.1) -- cycle ;
\draw (0,2.05) -- (4,2.05) -- (4,0.9) -- (0,0.9) -- cycle;
\end{tikzpicture}
\pause
\colb

{
\tt
\begin{tabular}{|lrcl|}\hline
\_\_main\_\_ & x & \arrow & 11 \\\hline\hline
foo & y & \arrow & 22 \\
    & z & \arrow & 33 \\\hline\hline
bar & x & \arrow & 66 \\
    & y & \arrow & 132 \\\hline
    \end{tabular}
    }

\bigskip
    
 Stack frames
\colc

\Show{\begin{itemize}
\item If you're in the box you can see it!
\item If you're not in the box you can't see it!
\end{itemize}
}
\end{frame}

\bfr{Fruitful functions}
\begin{lstlisting}
>>> math.sqrt(5)
2.2360679774997898
\end{lstlisting}

If you're in a script, the value is lost forever:

\begin{lstlisting}
math.sqrt(5)
\end{lstlisting}

You've got to do something with it:

\begin{lstlisting}
print(math.sqrt(5))
x = math.sqrt(5)
\end{lstlisting}


\end{frame}

\bfr{Functions that return nothing return \lstinline{None}}
\begin{lstlisting}
>>> x = print('hello')
hello
>>> print(x)
None
\end{lstlisting}
\end{frame}

\bfr{Defining your own fruitful functions}

\begin{lstlisting}
def smallest_digit(n):
    return n % 10
    
def average(x,y):
    return (x+y)/2

\end{lstlisting}
\end{frame}

\bfr{Functions are your friends}

\bi
\li Naming a chunk of code makes your program easier to read.
\li Eliminate redundancy.
\li You can debug one function at a time.
\li Errors give you a stack trace.
\li Functions can be used in many different programs.
\ei


\end{frame}

\bfr{Vocabulary}
\begin{description}
\li[function:]
A named sequence of statements that performs some useful operation. Functions may or may not take arguments and may or may not produce a result.
\li[function definition:]
A statement that creates a new function, specifying its name, parameters, and the statements it contains.
\li[function object:]
A value created by a function definition. The name of the function is a variable that refers to a function object.
\li[header:]
The first line of a function definition.
\li[body:]
The sequence of statements inside a function definition.
\end{description}
\end{frame}

\bfr{Vocabulary}
\begin{description}
\li[parameter:]
A name used inside a function to refer to the value passed as an argument.
\li[function call:]
A statement that runs a function. It consists of the function name followed by an argument list in parentheses.
\li[argument:]
A value provided to a function when the function is called. This value is assigned to the corresponding parameter in the function.
\li[local variable:]
A variable defined inside a function. A local variable can only be used inside its function.
\li[return value:]
The result of a function. If a function call is used as an expression, the return value is the value of the expression.
\end{description}
\end{frame}

\bfr{Vocabulary}
\begin{description}
\li[fruitful function:]
A function that returns a value.
\li[void function:]
A function that always returns None.
\li[None:]
A special value returned by void functions.
\li[module:]
A file that contains a collection of related functions and other definitions.
\li[import statement:]
A statement that reads a module file and creates a module object.
\li[module object:]
A value created by an import statement that provides access to the values defined in a module.
\li[dot notation:]
The syntax for calling a function in another module by specifying the module name followed by a dot (period) and the function name.
\end{description}
\end{frame}

\bfr{Vocabulary}
\begin{description}
\li[composition:]
Using an expression as part of a larger expression, or a statement as part of a larger statement.
\li[flow of execution:]
The order statements run in.
\li[stack diagram:]
A graphical representation of a stack of functions, their variables, and the values they refer to.
\li[frame:]
A box in a stack diagram that represents a function call. It contains the local variables and parameters of the function.
\li[traceback:]
A list of the functions that are executing, printed when an exception occurs.
\end{description}
\end{frame}




\end{document}
