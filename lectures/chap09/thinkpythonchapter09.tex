\documentclass{beamer}
\usetheme{Singapore}
\usepackage{changepage}

%\usepackage{pstricks,pst-node,pst-tree}
\usepackage{amssymb,latexsym}
\usepackage{tikz}
\usepackage{graphicx}
\usepackage{fancyvrb}
\usepackage{hyperref}
\usepackage{fancybox}
\usepackage[listings]{tcolorbox}

\definecolor{codegreen}{rgb}{0,0.6,0}
\definecolor{codegray}{rgb}{0.5,0.5,0.5}
\definecolor{codepurple}{rgb}{0.58,0,0.82}
\definecolor{backcolour}{rgb}{0.95,0.95,0.92}

\lstdefinestyle{mystyle}{
    language=Python,
    backgroundcolor=\color{backcolour},   
    commentstyle=\color{codegreen},
    keywordstyle=\color{magenta},
    numberstyle=\tiny\color{codegray},
    stringstyle=\color{codepurple},
    basicstyle=\ttfamily\footnotesize,
    breakatwhitespace=false,         
    breaklines=true,                 
    captionpos=b,                    
    keepspaces=true,                 
    numbers=left,                    
    numbersep=5pt,                  
    showspaces=false,                
    showstringspaces=false,
    showtabs=false,                  
    tabsize=2,
    escapechar=|,
    frame=single
}

\lstset{style=mystyle}


\newcommand{\bi}{\begin{itemize}}
\newcommand{\li}{\item}
\newcommand{\ei}{\end{itemize}}
\newcommand{\Show}[1]{
\begin{center}
\shadowbox{\begin{minipage}{0.8\textwidth}
          #1
          \end{minipage}}
\end{center}
}
\newcommand{\arrow}{\ensuremath{\rightarrow}}

\newcommand{\uparr}{\ensuremath{\uparrow}}


\newcommand{\fig}[2]{\centerline{\includegraphics[width=#1\textwidth]{#2}}}

\newcommand{\bfr}[1]{\begin{frame}[fragile]\frametitle{{ #1 }}}
\newcommand{\efr}{\end{frame}}

\newcommand{\cola}{\begin{columns}\begin{column}{0.5\textwidth}}
\newcommand{\colb}{\end{column}\begin{column}{0.5\textwidth}}
\newcommand{\colc}{\end{column}\end{columns}}


\title{Think Python 2e, Chapter 9 Notes}
\author{Case study: word play}

\begin{document}

\begin{frame}
\maketitle
\end{frame}

\bfr{Words}
\bi
\li The exercises for this chapter need a list of 113,783 English words, {\tt words.txt}
\li Available from:
\bi
\li The lecture folder for this chapter on github.
\li \url{http://thinkpython2.com/code/words.txt}
\ei
\ei
\cola
\begin{lstlisting}
aa
aah
aahed
aahing
aahs
aal
aalii
aaliis
aals
aardvark
...
\end{lstlisting}
\colb
\begin{lstlisting}
...
zymogene
zymogenes
zymogens
zymologies
zymology
zymoses
zymosis
zymotic
zymurgies
zymurgy
\end{lstlisting}
\colc

\end{frame}

\bfr{Process a file line by line}

\cola
\begin{lstlisting}
aa
aah
aahed
aahing
aahs
aal
aalii
aaliis
aals
aardvark
...
\end{lstlisting}
\colb
\begin{lstlisting}
>>> fin = open('words.txt')
>>> fin.readline()
'aa\n'
>>> fin.readline()
'aah\n'
>>> fin.readline().strip()
'aahed'
\end{lstlisting}
\colc
\end{frame}

\bfr{Process a file line by line}

\cola
\begin{lstlisting}
aa
aah
aahed
aahing
aahs
aal
aalii
aaliis
aals
aardvark
...
\end{lstlisting}
\colb
\begin{lstlisting}
fin = open('words.txt')
for line in fin:
    word = line.strip()
    print(word)
\end{lstlisting}
\pause
\begin{lstlisting}
fin = open('words.txt')
for line in fin:
    word = line.strip()
    if len(word) > 20:
        print(word)
\end{lstlisting}
\colc
\pause
\begin{center}
See text exercises
\end{center}

\end{frame}

\bfr{Reduction to a previously solved problem}
\begin{lstlisting}
def uses_only(word, available):
    for letter in word: 
        if letter not in available:
            return False
    return True
\end{lstlisting}
\begin{lstlisting}
def uses_all(word, required):
    for letter in required: 
        if letter not in word:
            return False
    return True
\end{lstlisting}
\pause
\begin{lstlisting}
def uses_all(word, required):
    return uses_only(required, word)
\end{lstlisting}
\end{frame}

\bfr{\tt is\_abecedarian}
Use a {\tt for} loop
\pause
\begin{lstlisting}
def is_abecedarian(word):
    previous = word[0]
    for c in word:
        if c < previous:
            return False
        previous = c
    return True
\end{lstlisting}
\pause
{\tt for} with {\tt range}
\begin{lstlisting}
def is_abecedarian(word):
    for i in range(len(word)-1):
        if word[i+1] < word[i]:
            return False
    return True
\end{lstlisting}
\end{frame}

\bfr{\tt is\_abecedarian}
Use recursion
\pause
\begin{lstlisting}
def is_abecedarian(word):
    if len(word) <= 1:
        return True
    if word[0] > word[1]:
        return False
    return is_abecedarian(word[1:])
\end{lstlisting}
\end{frame}

\bfr{\tt is\_abecedarian}
Use {\tt while} loop
\pause
\begin{lstlisting}
def is_abecedarian(word):
    i = 0
    while i < len(word)-1:
        if word[i+1] < word[i]:
            return False
        i = i+1
    return True
\end{lstlisting}
\end{frame}

\bfr{{\tt is\_palindrome} }

\begin{lstlisting}
def is_palindrome(word):
    i = 0
    j = len(word)-1

    while i<j:
        if word[i] != word[j]:
            return False
        i = i+1
        j = j-1

    return True
\end{lstlisting}
\pause
Reduced to a previous problem
\begin{lstlisting}
def is_palindrome(word):
    return is_reverse(word, word)
\end{lstlisting}
\end{frame}

\bfr{Special cases}
Testing \lstinline{has_no_e}
\bi
\li Test words with {\tt e}
\li Test words with no {\tt e}
\li Test words with {\tt e} at the beginning
\li Test words with {\tt e} at the end
\li Test words with {\tt e} in the middle
\li Test very long words
\li Test very short words, like {\tt e}
\li Test very short words, like the empty string
\ei
\end{frame}

\bfr{Testing}


Program testing can be used to show the presence of bugs, but never to show their absence!

\sl
\hfill — Edsger W. Dijkstra



\end{frame}


\bfr{Vocabulary}
\begin{description}
\li[file object:]
A value that represents an open file.
\li[reduction to a previously solved problem:]
A way of solving a problem by expressing it as an instance of a previously solved problem.
\li[special case:]
A test case that is atypical or non-obvious (and less likely to be handled correctly).
\end{description}
\end{frame}
\end{document}
