\documentclass{beamer}
\usetheme{Singapore}

%\usepackage{pstricks,pst-node,pst-tree}
\usepackage{amssymb,latexsym}
\usepackage{graphicx}
\usepackage{fancyvrb}
\usepackage{hyperref}
\usepackage{fancybox}
\usepackage{listings}
\definecolor{codegreen}{rgb}{0,0.6,0}
\definecolor{codegray}{rgb}{0.5,0.5,0.5}
\definecolor{codepurple}{rgb}{0.58,0,0.82}
\definecolor{backcolour}{rgb}{0.95,0.95,0.92}

\lstdefinestyle{mystyle}{
    language=Python,
    backgroundcolor=\color{backcolour},   
    commentstyle=\color{codegreen},
    keywordstyle=\color{magenta},
    numberstyle=\tiny\color{codegray},
    stringstyle=\color{codepurple},
    basicstyle=\ttfamily\footnotesize,
    breakatwhitespace=false,         
    breaklines=true,                 
    captionpos=b,                    
    keepspaces=true,                 
    numbers=left,                    
    numbersep=5pt,                  
    showspaces=false,                
    showstringspaces=false,
    showtabs=false,                  
    tabsize=2,
    escapechar=|
}

\lstset{style=mystyle,language=python}


\newcommand{\bi}{\begin{itemize}}
\newcommand{\li}{\item}
\newcommand{\ei}{\end{itemize}}
\newcommand{\Show}[1]{
\begin{center}
\shadowbox{\begin{minipage}{0.8\textwidth}
          #1
          \end{minipage}}
\end{center}
}
\newcommand{\arrow}{{\ensuremath{\longrightarrow}}}


\newcommand{\img}[2]{\centerline{\includegraphics[width=#1\textwidth]{#2}}}

\newcommand{\bfr}[1]{\begin{frame}[fragile]\frametitle{{ #1 }}}
\newcommand{\efr}{\end{frame}}

\newcommand{\cola}{\begin{columns}\begin{column}{0.5\textwidth}}
\newcommand{\colb}{\end{column}\begin{column}{0.5\textwidth}}
\newcommand{\colc}{\end{column}\end{columns}}


\title{Think Python 2e, Chapter 2 Notes}
\author{Geoffrey Matthews}

\begin{document}
\begin{frame}
\maketitle
\end{frame}

\bfr{Assignment statements}

\begin{lstlisting}
message = 'Hello world!'
n = 14 + 3
pi = 3.14169265
\end{lstlisting}

\bi
\li Variables name places in computer memory.
\li An assignment statement places data at the place 
represented by the variable.
\li An assignment statement changes the {\bf state}
of the computer.
\li We represent the state of a computer at a particular
time with a {\bf state diagram:}

\begin{minipage}{\textwidth}
\tt
\begin{tabular}{|rcl|}\hline
message & \arrow & 'Hello world!'\\
n & \arrow & 17 \\
pi &\arrow & 3.14159265\\\hline
\end{tabular}
\end{minipage}

\ei

\end{frame}
\bfr{Variable name syntax}

\begin{lstlisting}
>>> 76trombones = 'big parade'
SyntaxError: invalid syntax
>>> more@ = 1000000
SyntaxError: invalid syntax
>>> class = 'Advanced Theoretical Zymurgy'
SyntaxError: invalid syntax

\end{lstlisting}


\end{frame}

\bfr{Python keywords}
\begin{lstlisting}
False      class      finally    is         return
None       continue   for        lambda     try
True       def        from       nonlocal   while
and        del        global     not        with
as         elif       if         or         yield
assert     else       import     pass
break      except     in         raise
\end{lstlisting}


\end{frame}

\bfr{Expressions {\em vs.} Statements}

Expressions:
\begin{lstlisting}
42
n
n + 25
'hello'
\end{lstlisting}

\vfill

Statements:
\begin{lstlisting}
n = 17
print(n)
print('hello')
print(14 + 3)
n = n * 2
\end{lstlisting}
\vfill


\end{frame}
\bfr{String operations}


\begin{lstlisting}
>>> first = 'throat'
>>> second = 'warbler'
>>> first + second
throatwarbler
>>> 'spam' * 3
spamspamspam
\end{lstlisting}


\end{frame}

\bfr{Comments}


\begin{lstlisting}
# compute the percentage of the hour that has elapsed
percentage = (minute*100) / 60
\end{lstlisting}




\begin{lstlisting}
percentage = (minute*100) / 60 # percentage of an hour
\end{lstlisting}

\end{frame}
\bfr{Good and Bad Comments}

Good comment:

\begin{lstlisting}
v = 5     # velocity in meters/second. 

\end{lstlisting}

Bad comment:
\begin{lstlisting}
v = 5     # assign 5 to v
\end{lstlisting}


\end{frame}

\bfr{Debugging}
\begin{description}
\li[Syntax error:]  These are found by Python when you try to 
run the program, or use the check module button.
\li[Runtime error:]  These errors are not found by Python
until the program is running.  An example would be
dividing by zero or taking the square root of a negative
number.  Also called {\bf exceptions}.  We say that the
running program raised an exception.
\li[Semantic error:]  An error in telling the computer
what you mean.
For example, if you wanted the speed in kilometers
per hour, but your instructions calculated
speed in miles per hour, the program would compute
an incorrect result and give you the wrong answer
 with no warning.
\end{description}

\Show{The computer {\em always} does exactly what
you say, but only {\em sometimes} does what you mean.}

\end{frame}

\bfr{Vocabulary}
\begin{description}
\li[variable:]
A name that refers to a value.
\li[assignment:]
A statement that assigns a value to a variable.
\li[state diagram:]
A graphical representation of a set of variables and the values they refer to.
\li[keyword:]
A reserved word that is used to parse a program; you cannot use keywords like if, def, and while as variable names.
\li[operand:]
One of the values on which an operator operates.
\end{description}
\end{frame}

\bfr{Vocabulary}
\begin{description}
\li[expression:]
A combination of variables, operators, and values that represents a single result.
\li[evaluate:]
To simplify an expression by performing the operations in order to yield a single value.
\li[statement:]
A section of code that represents a command or action. So far, the statements we have seen are assignments and print statements.
\li[execute:]
To run a statement and do what it says.
\end{description}
\end{frame}

\bfr{Vocabulary}
\begin{description}
\li[interactive mode:]
A way of using the Python interpreter by typing code at the prompt.
\li[script mode:]
A way of using the Python interpreter to read code from a script and run it.
\li[script:]
A program stored in a file.
\end{description}
\end{frame}

\bfr{Vocabulary}
\begin{description}
\li[comment:]
Information in a program that is meant for other programmers (or anyone reading the source code) and has no effect on the execution of the program.
\li[syntax error:]
An error in a program that makes it impossible to parse (and therefore impossible to interpret).
\li[exception:]
An error that is detected while the program is running.
\li[semantics:]
The meaning of a program.
\li[semantic error:]
An error in a program that makes it do something other than what the programmer intended.
\end{description}
\end{frame}



\end{document}
