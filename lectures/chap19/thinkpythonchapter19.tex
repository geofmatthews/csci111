\documentclass{beamer}
\usetheme{Singapore}
\usepackage{changepage}

%\usepackage{pstricks,pst-node,pst-tree}
\usepackage{amssymb,latexsym}
\usepackage{tikz}
\usepackage{graphicx}
\usepackage{fancyvrb}
\usepackage{hyperref}
\usepackage{fancybox}
\usepackage[listings]{tcolorbox}

\definecolor{codegreen}{rgb}{0,0.6,0}
\definecolor{codegray}{rgb}{0.5,0.5,0.5}
\definecolor{codepurple}{rgb}{0.58,0,0.82}
\definecolor{backcolour}{rgb}{0.95,0.95,0.92}

\lstdefinestyle{mystyle}{
    language=Python,
    backgroundcolor=\color{backcolour},   
    commentstyle=\color{codegreen},
    keywordstyle=\color{magenta},
    numberstyle=\tiny\color{codegray},
    stringstyle=\color{codepurple},
    basicstyle=\ttfamily\footnotesize,
    breakatwhitespace=false,         
    breaklines=true,                 
    captionpos=b,                    
    keepspaces=true,                 
    numbers=left,                    
    numbersep=5pt,                  
    showspaces=false,                
    showstringspaces=false,
    showtabs=false,                  
    tabsize=2,
    escapechar=|,
    frame=single
}

\lstset{style=mystyle}


\newcommand{\bi}{\begin{itemize}}
\newcommand{\li}{\item}
\newcommand{\ei}{\end{itemize}}
\newcommand{\Show}[1]{
\begin{center}
\shadowbox{\begin{minipage}{0.8\textwidth}
          #1
          \end{minipage}}
\end{center}
}
\newcommand{\arrow}{\ensuremath{\rightarrow}}

\newcommand{\uparr}{\ensuremath{\uparrow}}


\newcommand{\fig}[2]{\centerline{\includegraphics[width=#1\textwidth]{#2}}}

\newcommand{\bfr}[1]{\begin{frame}[fragile]\frametitle{{ #1 }}}
\newcommand{\efr}{\end{frame}}

\newcommand{\cola}{\begin{columns}\begin{column}{0.5\textwidth}}
\newcommand{\colb}{\end{column}\begin{column}{0.5\textwidth}}
\newcommand{\colc}{\end{column}\end{columns}}


\title{Think Python 2e, Chapter 19 Notes}
\author{Goodies: Conditional expressions and list comprehensions}

\begin{document}

\begin{frame}
\maketitle
\end{frame}

\bfr{Conditional expressions}
\begin{lstlisting}
if x > 0:
    y = math.log(x)
else:
    y = float('nan')
\end{lstlisting}
\begin{lstlisting}
y = math.log(x) if x > 0 else float('nan')
\end{lstlisting}
\end{frame}

\bfr{Recursive functions using conditional expressions}
\begin{lstlisting}
def factorial(n):
    if n == 0:
        return 1
    else:
        return n * factorial(n-1)
\end{lstlisting}
\begin{lstlisting}
def factorial(n):
    return 1 if n == 0 else n * factorial(n-1)
\end{lstlisting}
\end{frame}

\bfr{Optional arguments using conditional expressions}
\begin{lstlisting}
    def __init__(self, name, contents=None):
        self.name = name
        if contents == None:
            contents = []
        self.pouch_contents = contents
\end{lstlisting}
\begin{lstlisting}
    def __init__(self, name, contents=None):
        self.name = name
        self.pouch_contents = [] if contents == None else contents 
\end{lstlisting}
\end{frame}

\bfr{List comprehensions}
\begin{lstlisting}
def capitalize_all(t):
    res = []
    for s in t:
        res.append(s.capitalize())
    return res
\end{lstlisting}
\begin{lstlisting}
def capitalize_all(t):
    return [s.capitalize() for s in t]
\end{lstlisting}
\end{frame}

\bfr{List comprehensions}
\begin{lstlisting}
def only_upper(t):
    res = []
    for s in t:
        if s.isupper():
            res.append(s)
    return res
\end{lstlisting}
\begin{lstlisting}
def only_upper(t):
    return [s for s in t if s.isupper()]
\end{lstlisting}
\end{frame}

\bfr{Debugging list comprehensions}
\bi
\li Cannot put a print statement in the loop.
\li Recommended only if simple or already done in loop
\ei
\end{frame}

\end{document}
