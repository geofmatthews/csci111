\documentclass{beamer}
\usetheme{Singapore}

%\usepackage{pstricks,pst-node,pst-tree}
\usepackage{amssymb,latexsym}
\usepackage{tikz}
\usepackage{graphicx}
\usepackage{fancyvrb}
\usepackage{hyperref}
\usepackage{fancybox}
\usepackage[listings]{tcolorbox}

\definecolor{codegreen}{rgb}{0,0.6,0}
\definecolor{codegray}{rgb}{0.5,0.5,0.5}
\definecolor{codepurple}{rgb}{0.58,0,0.82}
\definecolor{backcolour}{rgb}{0.95,0.95,0.92}

\lstdefinestyle{mystyle}{
    language=Python,
    backgroundcolor=\color{backcolour},   
    commentstyle=\color{codegreen},
    keywordstyle=\color{magenta},
    numberstyle=\tiny\color{codegray},
    stringstyle=\color{codepurple},
    basicstyle=\ttfamily\footnotesize,
    breakatwhitespace=false,         
    breaklines=true,                 
    captionpos=b,                    
    keepspaces=true,                 
    numbers=left,                    
    numbersep=5pt,                  
    showspaces=false,                
    showstringspaces=false,
    showtabs=false,                  
    tabsize=2,
    escapechar=|,
    frame=single
}

\lstset{style=mystyle}


\newcommand{\bi}{\begin{itemize}}
\newcommand{\li}{\item}
\newcommand{\ei}{\end{itemize}}
\newcommand{\Show}[1]{
\begin{center}
\shadowbox{\begin{minipage}{0.8\textwidth}
          #1
          \end{minipage}}
\end{center}
}
\newcommand{\arrow}{\ensuremath{\rightarrow}}


\newcommand{\img}[2]{\centerline{\includegraphics[width=#1\textwidth]{#2}}}

\newcommand{\bfr}[1]{\begin{frame}[fragile]\frametitle{{ #1 }}}
\newcommand{\efr}{\end{frame}}

\newcommand{\cola}{\begin{columns}\begin{column}{0.5\textwidth}}
\newcommand{\colb}{\end{column}\begin{column}{0.5\textwidth}}
\newcommand{\colc}{\end{column}\end{columns}}


\title{Think Python 2e, Chapter 4 Notes}
\author{Geoffrey Matthews}

\begin{document}

\begin{frame}
\maketitle

\end{frame}

\bfr{The turtle module}
\begin{lstlisting}
import turtle
bob = turtle.Turtle()
turtle.mainloop()
\end{lstlisting}

\bi
\li This will create a single turtle, \lstinline{bob}
\li \lstinline{bob} will react to commands entered in the shell
\li Try:
\begin{lstlisting}
bob.fd(100)
bob.rt(120)
bob.fd(100)
bob.rt(120)
bob.fd(100)
\end{lstlisting}
\ei


\end{frame}

\bfr{Simple looping}
What do you think this does?
\begin{lstlisting})
for i in range(4):
    print('Hello!')
\end{lstlisting}

\end{frame}


\bfr{Simple looping}
What do you think this does?
\begin{lstlisting}
import turtle
bob = turtle.Turtle()

for i in range(4):
    bob.fd(100)
    bob.lt(90)
\end{lstlisting}

\end{frame}

\bfr{Encapsulation}
Write a {\bf function} that takes a turtle as
argument, and makes it draw a square.
\pause
\begin{lstlisting}
def square(t):
    for i in range(4):
        t.fd(100)
        t.lt(90)
\end{lstlisting}
\begin{lstlisting}  
square(bob)
square(alice)
\end{lstlisting}
\bi
\li
Also called {\bf Abstraction}: giving a name to something.
\li
Makes it very easy to reuse code in different contexts.
\ei

\end{frame}



\bfr{Generalization}
Add a parameter \lstinline{length} to \lstinline{square}
\cola
\begin{lstlisting}
def square(t):
    for i in range(4):
        t.fd(100)
        t.lt(90)
\end{lstlisting}
\begin{lstlisting}  
square(bob)
square(alice)
\end{lstlisting}
\colb
\pause
\begin{lstlisting}
def square(t, length):
    for i in range(4):
        t.fd(length)
        t.lt(90)
\end{lstlisting}
\begin{lstlisting}  
square(bob, 100)
square(alice, 200)
\end{lstlisting}
\colc

\end{frame}


\bfr{Further Generalization}
Change \lstinline{square} to \lstinline{polygon}.
\cola
\begin{lstlisting}
def square(t, length):
    for i in range(4):
        t.fd(length)
        t.lt(90)
\end{lstlisting}
\begin{lstlisting}  
square(bob, 100)
square(alice, 200)
\end{lstlisting}
\colb
\pause
\begin{lstlisting}
def polygon(t, n, length):
    angle = 360/n
    for i in range(n):
        t.fd(length)
        t.lt(angle)
\end{lstlisting}
\begin{lstlisting}  
polygon(bob, 5, 100)
polygon(alice, 8, 200)
\end{lstlisting}
\colc

\end{frame}

\bfr{Keyword arguments}

\begin{lstlisting}
def polygon(t, n, length):
    angle = 360/n
    for i in range(n):
        t.fd(length)
        t.lt(angle)
\end{lstlisting}
\begin{lstlisting}  
polygon(bob, 5, 100)
polygon(alice, 8, 200)
\end{lstlisting}
\begin{lstlisting}  
polygon(bob, n=5, length=100)
polygon(alice, length=200, n=8)
\end{lstlisting}

\end{frame}

\bfr{Interface design}
\begin{lstlisting}
def circle(t, r):
    circumference = 2*math.pi*r
    n = max(int(circumference/5), 5)
    polygon(t, n, circumference/n)
\end{lstlisting}
\bi
\li We want to draw a {\em smooth} circle using a 
polygon with many sides.
\li How many sides do we use? 
\li This should {\em not} be part of the interface.
\li It is part of the {\em implementation}, but should not concern the user.
\li We find a number so that the straight lines are
not more than 5 pixels in length.
\ei

\end{frame}

\bfr{Refactoring}
We want to write \verb|arc| that will draw
part of a circle. 

\begin{lstlisting}
def polygon(t, n, length):
    angle = 360/n
    for i in range(n):
        t.fd(length)
        t.lt(angle)
\end{lstlisting}
\begin{lstlisting}
def arc(t, r, angle):
    arc_length = 2 * math.pi * r * angle / 360
    n = max(int(arc_length / 3), 1)
    step_length = arc_length / n
    step_angle = angle / n    
    for i in range(n):
        t.fd(step_length)
        t.lt(step_angle)
\end{lstlisting}

The last part of this function looks like \verb|polygon|
but we can't reuse \verb|polygon| since it assumes
we want $360^\circ$

\end{frame}

\bfr{Refactoring}
Old definition:
\begin{lstlisting}
def polygon(t, n, length):
    angle = 360/n
    for i in range(n):
        t.fd(length)
        t.lt(angle)
\end{lstlisting}
New definition:
\begin{lstlisting}
def polyline(t, n, length, angle):
    for i in range(n):
        t.fd(length)
        t.lt(angle)        
def polygon(t, n, length):
    angle = 360.0 / n
    polyline(t, n, length, angle)
\end{lstlisting}

We broke one function into two, like factoring $15 = 3\cdot 5$
\end{frame}

\bfr{Refactoring}
\begin{lstlisting}
def polyline(t, n, length, angle):
    for i in range(n):
        t.fd(length)
        t.lt(angle)        
def polygon(t, n, length):
    angle = 360.0 / n
    polyline(t, n, length, angle)
\end{lstlisting}
New version is more useful:
\begin{lstlisting}
def arc(t, r, angle):
    arc_length = 2 * math.pi * r * angle / 360
    n = int(arc_length / 3) + 1
    step_length = arc_length / n
    step_angle = float(angle) / n
    polyline(t, n, step_length, step_angle)
def circle(t, r):
    arc(t, r, 360)
\end{lstlisting}
\end{frame}

\bfr{A development plan}

\begin{enumerate}
\li
Start by writing a small program with no function definitions.
\li
Once you get the program working, identify a coherent piece of it, encapsulate the piece in a function and give it a name.
\li
Generalize the function by adding appropriate parameters.
\li
Repeat steps 1–3 until you have a set of working functions.
\li
Look for opportunities to improve the program by refactoring.
\bi\li For example, if you have similar code in several places, consider factoring it into an appropriately general function.
\ei
\end{enumerate}
\end{frame}

\bfr{Docstrings}
\begin{lstlisting}
def polyline(t, n, length, angle):
    """Draws n line segments with the given length and
    angle (in degrees) between them.  t is a turtle.
    """    
    for i in range(n):
        t.fd(length)
        t.lt(angle)
\end{lstlisting}
\begin{lstlisting}
help(polyline)
Help on function polyline in module __main__:

polyline(t, n, length, angle)
    Draws n line segments with the given length and
    angle (in degrees) between them.  t is a turtle.
\end{lstlisting}

\end{frame}

\bfr{Documentation and debugging}
\bi
\li Functions expect their arguments to meet certain conditions.
\bi
\li For example, \verb|polyline| requires four arguments: \verb|t|
 has to be a Turtle; \verb|n| has to be an integer; 
 \verb|length| should be a positive number; 
 and \verb|angle| has to be a number, 
 which is understood to be in degrees.
\ei
\li These expectations are called the {\bf preconditions}
\li If the preconditions are met, the function
should guarantee that certain other conditions are also met.
\bi\li For example \verb|polyline| will draw line segments
of the right length and angle.\ei
\li These expectactions are called the {\bf postconditions}

\ei
\end{frame}

\bfr{Documentation and debugging}
\bi
\li If your program has a bug, and the preconditions
for a function are not met, the bug is probably
in the code that calls the function.
\li If the preconditions are met, but the postconditions
are not, then the bug is in the function.
\li If the preconditions and postconditions
of a function are clear, they can help with debugging.
\ei

\end{frame}
\bfr{Vocabulary}
\begin{description}
\li[method:]
A function that is associated with an object and called using dot notation.
\li[loop:]
A part of a program that can run repeatedly.
\li[encapsulation:]
The process of transforming a sequence of statements into a function definition.
\li[generalization:]
The process of replacing something unnecessarily specific (like a number) with something appropriately general (like a variable or parameter).
\li[keyword argument:]
An argument that includes the name of the parameter as a “keyword”.
\li[interface:]
A description of how to use a function, including the name and descriptions of the arguments and return value.
\end{description}
\end{frame}
\bfr{Vocabulary}
\begin{description}
\li[refactoring:]
The process of modifying a working program to improve function interfaces and other qualities of the code.
\li[development plan:]
A process for writing programs.
\li[docstring:]
A string that appears at the top of a function definition to document the function’s interface.
\li[precondition:]
A requirement that should be satisfied by the caller before a function starts.
\li[postcondition:]
A requirement that should be satisfied by the function before it ends.

\end{description}
\end{frame}

\end{document}
