\documentclass{beamer}
\usetheme{Singapore}
\usepackage{changepage}

%\usepackage{pstricks,pst-node,pst-tree}
\usepackage{amssymb,latexsym}
\usepackage{tikz}
\usepackage{graphicx}
\usepackage{fancyvrb}
\usepackage{hyperref}
\usepackage{fancybox}
\usepackage[listings]{tcolorbox}

\definecolor{codegreen}{rgb}{0,0.6,0}
\definecolor{codegray}{rgb}{0.5,0.5,0.5}
\definecolor{codepurple}{rgb}{0.58,0,0.82}
\definecolor{backcolour}{rgb}{0.95,0.95,0.92}

\lstdefinestyle{mystyle}{
    language=Python,
    backgroundcolor=\color{backcolour},   
    commentstyle=\color{codegreen},
    keywordstyle=\color{magenta},
    numberstyle=\tiny\color{codegray},
    stringstyle=\color{codepurple},
    basicstyle=\ttfamily\footnotesize,
    breakatwhitespace=false,         
    breaklines=true,                 
    captionpos=b,                    
    keepspaces=true,                 
    numbers=left,                    
    numbersep=5pt,                  
    showspaces=false,                
    showstringspaces=false,
    showtabs=false,                  
    tabsize=2,
    escapechar=|,
    frame=single
}

\lstset{style=mystyle}


\newcommand{\bi}{\begin{itemize}}
\newcommand{\li}{\item}
\newcommand{\ei}{\end{itemize}}
\newcommand{\Show}[1]{
\begin{center}
\shadowbox{\begin{minipage}{0.8\textwidth}
          #1
          \end{minipage}}
\end{center}
}
\newcommand{\arrow}{\ensuremath{\rightarrow}}

\newcommand{\uparr}{\ensuremath{\uparrow}}


\newcommand{\fig}[2]{\centerline{\includegraphics[width=#1\textwidth]{#2}}}

\newcommand{\bfr}[1]{\begin{frame}[fragile]\frametitle{{ #1 }}}
\newcommand{\efr}{\end{frame}}

\newcommand{\cola}{\begin{columns}\begin{column}{0.5\textwidth}}
\newcommand{\colb}{\end{column}\begin{column}{0.5\textwidth}}
\newcommand{\colc}{\end{column}\end{columns}}


\title{Think Python 2e, Chapter 17 Notes}
\author{Classes and Methods}

\begin{document}

\begin{frame}
\maketitle
\end{frame}



\bfr{Methods}
\bi
\li
Need tighter relationship between classes and the functions
that deal with them.
\li Methods are semantically the same as functions.
\li The syntax for methods is different from functions.
\li Methods are defined inside a class definition.
\li This makes the relation between class and method explicit.
\ei
\end{frame}

\bfr{{\bf NON} object-oriented way}
\begin{lstlisting}
class Time:
    """Represents the time of day."""

def print_time(time):
    print('%.2d:%.2d:%.2d' % 
                (time.hour, time.minute, time.second))
\end{lstlisting}
\begin{lstlisting}
>>> start = Time()
>>> start.hour = 9
>>> start.minute = 45
>>> start.second = 00
\end{lstlisting}
Only way to call the function:
\begin{lstlisting}
>>> print_time(start)
09:45:00
\end{lstlisting}
\end{frame}

\bfr{Object-oriented way}
\begin{lstlisting}
class Time:
    """Represents the time of day."""
    def print_time(time):
        print('%.2d:%.2d:%.2d' % 
            (time.hour, time.minute, time.second))
\end{lstlisting}
There are now two ways to call the function:
\begin{lstlisting}
>>> Time.print_time(start)
09:45:00
>>> start.print_time()
09:45:00
\end{lstlisting}
\bi
\li The second is more concise.
\li \lstinline{start} is the actual parameter bound to \lstinline{time}
\li \lstinline{start} is called the {\bf subject}
\ei
\end{frame}

\bfr{\tt self}
\begin{lstlisting}
class Time:
    def print_time(time):
        print('%.2d:%.2d:%.2d' % 
            (time.hour, time.minute, time.second))
\end{lstlisting}
\bi
\li By convention, the formal parameter is usually called \lstinline{self}
\ei
\begin{lstlisting}
class Time:
    def print_time(self):
        print('%.2d:%.2d:%.2d' % 
            (self.hour, self.minute, self.second))
\end{lstlisting}
\begin{lstlisting}
>>> start.print_time()
09:45:00
\end{lstlisting}
\end{frame}

\bfr{Function-oriented  {\em vs.} object-oriented programming}
Function is focus:
\begin{lstlisting}
>>> print_time(start)
09:45:00
\end{lstlisting}
Object is focus:
\begin{lstlisting}
>>> start.print_time()
09:45:00
\end{lstlisting}
\pause
\bi
\li Notice you can write \lstinline{time_to_int} as a method, but
not \lstinline{int_to_time}.
\li Why not?
\ei
\end{frame}


\bfr{\tt increment}
\begin{lstlisting}
# inside class Time:

    def increment(self, seconds):
        seconds += self.time_to_int()
        return int_to_time(seconds)
\end{lstlisting}
\bi\li This is a pure function\ei
\begin{lstlisting}
>>> start.print_time()
09:45:00
>>> end = start.increment(1337)
>>> end.print_time()
10:07:17
\end{lstlisting}
\bi
\li \lstinline{increment} is defined with two formal parameters
\li \lstinline{increment} is called with one subject and one actual parameter
\ei

\end{frame}

\bfr{Error message can be confusing}
\begin{lstlisting}
>>> end = start.increment(1337, 460)
TypeError: increment() takes 2 positional arguments but 3 were given
\end{lstlisting}
\bi
\li But I only gave two parameters!
\pause
\li Wrong!  You gave the subject and two parameters.
\li That's three
\ei
\end{frame}

\bfr{Positional arguments}

\bi
\li
A {\bf positional argument} is an argument that doesn't have a parameter
name; that is, it is not a keyword argument.
\ei
\begin{lstlisting}
sketch(parrot, cage, dead=True)
\end{lstlisting}
\bi
\li
\lstinline{parrot} and \lstinline{cage} are positional, and 
\lstinline{dead} is a keyword argument.

\ei

\end{frame}

\bfr{Methods with two objects}
\begin{lstlisting}
# inside class Time:

  def is_after(self, other):
    return self.time_to_int() > other.time_to_int()
\end{lstlisting}
\bi
\li
\lstinline{self} and \lstinline{other} are conventional names.
\ei
\begin{lstlisting}
>>> end.is_after(start)
True
\end{lstlisting}


\end{frame}

\bfr{\tt \_\_init\_\_}
\begin{lstlisting}
# inside class Time:

    def __init__(self, hour=0, minute=0, second=0):
        self.hour = hour
        self.minute = minute
        self.second = second
\end{lstlisting}

\begin{lstlisting}
>>> time = Time(9, 45)
>>> time.print_time()
09:45:00
\end{lstlisting}


\end{frame}
\bfr{\tt \_\_str\_\_}
\begin{lstlisting}
# inside class Time:

    def __str__(self):
        return '%.2d:%.2d:%.2d' % 
                (self.hour, self.minute, self.second)
\end{lstlisting}

\begin{lstlisting}
>>> time = Time(9, 45)
>>> print(time)
09:45:00
\end{lstlisting}


\end{frame}

\bfr{Operator overloading}

\bi
\li
Every operator in Python has a dunder method to overload it.
\li
Here we overload addition, {\em i.e.} the \lstinline{+} operator.
\ei
\begin{lstlisting}
# inside class Time:

    def __add__(self, other):
        seconds = self.time_to_int() +
                  other.time_to_int()
        return int_to_time(seconds)
\end{lstlisting}

\begin{lstlisting}
>>> start = Time(9, 45)
>>> duration = Time(1, 35)
>>> print(start + duration)
11:20:00
\end{lstlisting}


\end{frame}

\bfr{Type based dispatch}
\begin{lstlisting}
# inside class Time:

    def __add__(self, other):
        if isinstance(other, Time):
            return self.add_time(other)
        else:
            return self.increment(other)

    def add_time(self, other):
        seconds = self.time_to_int() + 
                  other.time_to_int()
        return int_to_time(seconds)

    def increment(self, seconds):
        seconds += self.time_to_int()
        return int_to_time(seconds)
\end{lstlisting}


\end{frame}

\bfr{Type based dispatch}
We can now use this as follows
\begin{lstlisting}
>>> start = Time(9, 45)
>>> duration = Time(1, 35)
>>> print(start + duration)
11:20:00
>>> print(start + 1337)
10:07:17
\end{lstlisting}
Unfortunately it is not commutative.  Solution:
\begin{lstlisting}
# inside class Time:

    def __radd__(self, other):
        return self.__add__(other)
\end{lstlisting}
\begin{lstlisting}
>>> print(1337 + start)
10:07:17
\end{lstlisting}

\end{frame}

\bfr{Polymorphism}

Functions that work with several types are {\bf polymorphic}

\begin{lstlisting}
def histogram(s):
    d = dict()
    for c in s:
        if c not in d:
            d[c] = 1
        else:
            d[c] = d[c]+1
    return d
\end{lstlisting}

\begin{lstlisting}
>>> t = ['spam', 'egg', 'spam', 'spam', 'bacon', 'spam']
>>> histogram(t)
{'bacon': 1, 'egg': 1, 'spam': 4}
>>> histogram('banana')
{'b': 1, 'a': 3, 'n': 2}
\end{lstlisting}

\end{frame}

\bfr{Polymorphism and code reuse}
\lstinline{sum} will work with any items
which support addition
\begin{lstlisting}>>> t1 = Time(7, 43)
>>> t2 = Time(7, 41)
>>> t3 = Time(7, 37)
>>> total = sum([t1, t2, t3])
>>> print(total)
23:01:00
\end{lstlisting}

Polymorphism frequently surprises us and works
for types we didn't even know they would.

\end{frame}

\bfr{Debugging}
\bi
\li
It is legal to add attributes anywhere in the execution
of a program.
\li 
It is still a bad idea:  same types should have same attributes.
\li
Add attributes only in the \lstinline{__init__} method.
\li
If you have to check if an object has an attribute use \lstinline{hasattr}
\li
Can also use \lstinline{vars} which takes an object
and returns a dictionary mapping from attribute names to values.
\li
\lstinline{getattr} takes an object and an attribute name
and returns the attribute's value.
\begin{lstlisting}
item.x == getattr(item, 'x')
\end{lstlisting}
\ei

\end{frame}

\bfr{Interface and Implementation}

\bi
\li
Separate interfaces from implementations
\li
Methods should not depend on attributes
\li Example:
\bi
\li \lstinline{Time} used attributes:
\lstinline{hour}, \lstinline{minute},  \lstinline{second}
\li Instead, it could have used just:
\lstinline{seconds}
\li Different methods are easier with different
representations.
\ei
\li The user of the interface should not know
the implementation.
\li You can change the implementation, to make it faster, smaller,
whatever, without changing the interface.
\li Code that uses the class should not change
when the implementation changes.
\ei


\end{frame}

\bfr{Vocabulary}
\begin{description}

\li[object-oriented language:]
A language that provides features, such as programmer-defined types and methods, that facilitate object-oriented programming.
\li[object-oriented programming:]
A style of programming in which data and the operations that manipulate it are organized into classes and methods.
\li[method:]
A function that is defined inside a class definition and is invoked on instances of that class.
\li[subject:]
The object a method is invoked on.
\end{description}
\end{frame}

\bfr{Random Warning!}

\centerline{\shadowbox{Don't initialize objects with mutables!}}



\end{frame}

\bfr{Vocabulary}
\begin{description}

\li[positional argument:]
An argument that does not include a parameter name, so it is not a keyword argument.
\li[operator overloading:]
Changing the behavior of an operator like + so it works with a programmer-defined type.
\li[type-based dispatch:]
A programming pattern that checks the type of an operand and invokes different functions for different types.
\li[polymorphic:]
Pertaining to a function that can work with more than one type.
\end{description}
\end{frame}
\end{document}
