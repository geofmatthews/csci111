\documentclass{beamer}
\usetheme{Singapore}

%\usepackage{pstricks,pst-node,pst-tree}
\usepackage{amssymb,latexsym}
\usepackage{tikz}
\usepackage{graphicx}
\usepackage{fancyvrb}
\usepackage{hyperref}
\usepackage{fancybox}
\usepackage[listings]{tcolorbox}

\definecolor{codegreen}{rgb}{0,0.6,0}
\definecolor{codegray}{rgb}{0.5,0.5,0.5}
\definecolor{codepurple}{rgb}{0.58,0,0.82}
\definecolor{backcolour}{rgb}{0.95,0.95,0.92}

\lstdefinestyle{mystyle}{
    language=Python,
    backgroundcolor=\color{backcolour},   
    commentstyle=\color{codegreen},
    keywordstyle=\color{magenta},
    numberstyle=\tiny\color{codegray},
    stringstyle=\color{codepurple},
    basicstyle=\ttfamily\footnotesize,
    breakatwhitespace=false,         
    breaklines=true,                 
    captionpos=b,                    
    keepspaces=true,                 
    numbers=left,                    
    numbersep=5pt,                  
    showspaces=false,                
    showstringspaces=false,
    showtabs=false,                  
    tabsize=2,
    escapechar=|,
    frame=single
}

\lstset{style=mystyle}


\newcommand{\bi}{\begin{itemize}}
\newcommand{\li}{\item}
\newcommand{\ei}{\end{itemize}}
\newcommand{\Show}[1]{
\begin{center}
\shadowbox{\begin{minipage}{0.8\textwidth}
          #1
          \end{minipage}}
\end{center}
}
\newcommand{\arrow}{\ensuremath{\rightarrow}}


\newcommand{\img}[2]{\centerline{\includegraphics[width=#1\textwidth]{#2}}}

\newcommand{\bfr}[1]{\begin{frame}[fragile]\frametitle{{ #1 }}}
\newcommand{\efr}{\end{frame}}

\newcommand{\cola}{\begin{columns}\begin{column}{0.5\textwidth}}
\newcommand{\colb}{\end{column}\begin{column}{0.5\textwidth}}
\newcommand{\colc}{\end{column}\end{columns}}


\title{Think Python 2e, Chapter 5 Notes}
\author{Geoffrey Matthews}

\begin{document}

\begin{frame}
\maketitle

\end{frame}

\bfr{Three kinds of division}
\begin{lstlisting}
>>> 13 / 5
2.6
>>> 13 // 5
2
>>> 13 % 5
3
\end{lstlisting}

\bi
\li
Integer division is good for getting hours and minutes from minutes.
\li Pints and gallons, feet and inches, ...
\ei
\end{frame}

\bfr{Boolean expressions}
\bi
\li A new type of object:  boolean
\ei
\begin{lstlisting}
>>> 5 == 5
True
>>> 5 == 6
False
\end{lstlisting}

\begin{lstlisting}
>>> type(True)
<class 'bool'>
>>> type(False)
<class 'bool'>
\end{lstlisting}
\end{frame}

\bfr{Relational operators}

Relate two numbers

\begin{lstlisting}
x != y               # x is not equal to y
x > y                # x is greater than y
x < y                # x is less than y
x >= y               # x is greater than or equal to y
x <= y               # x is less than or equal to y
\end{lstlisting}


\end{frame}


\bfr{Logical operators} 

 Operate on one or two booleans 
 
\begin{lstlisting}
x and y         # true when both x and y are true
x or y          # true when x or y or both are true
not x           # true when x is not true
\end{lstlisting}

Many other things act truthy:
\begin{lstlisting}
>>> 33 and True
True
>>> 0 and True
0
>>> 'hello' and True
True
>>> '' and True
''
\end{lstlisting}

\end{frame}

\bfr{Conditional execution}
\begin{lstlisting}
if x > 0:
    print('x is positive')
\end{lstlisting}

\bi
\li The boolean expression is called the {\bf condition}
\li The indented part is called the {\bf body}
\li Any number of statements can be in the body
\li If you want zero statements in the body:
\begin{lstlisting}
if x > 0:
    pass
\end{lstlisting}
\ei

\end{frame}

\bfr{Alternative execution}
\begin{lstlisting}
if x % 2 == 0:
    print('x is even')
else:
    print('x is odd')
\end{lstlisting}

\bi
\li The two alternatives are called {\bf branches}
\li Only one branch will be executed
\ei

\end{frame}

\bfr{Chained conditionals}
\begin{lstlisting}
if x < y:
    print('x is less than y')
elif x > y:
    print('x is greater than y')
else:
    print('x and y are equal')
\end{lstlisting}

\bi
\li The alternatives are called {\bf branches}
\li If more than one condition is true only the first such
branch is executed.
\ei

\end{frame}
\bfr{Nested conditionals}

These are equivalent:
\begin{lstlisting}
if x == y:
    print('x and y are equal')
else:
    if x < y:
        print('x is less than y')
    else:
        print('x is greater than y')
\end{lstlisting}
\begin{lstlisting}
if x < y:
    print('x is less than y')
elif x > y:
    print('x is greater than y')
else:
    print('x and y are equal')
\end{lstlisting}

\bi
\li Generally the \verb|elif| form is clearer.
\ei

\end{frame}

\bfr{Circular definition}
\begin{description}
\item[vorpal:]  an adjective referring to any thing that is vorpal
\end{description}
\end{frame}

\bfr{Recursive definition}
\begin{description}
\item[vorpal:]  
\begin{itemize}
\item a cat
\item a dog
\item a box containing something that is vorpal
\end{itemize}
\end{description}
\end{frame}

\bfr{Recursion}
\cola
\begin{lstlisting}
def countdown(n):
    if n <= 0:
        print('Blastoff!')
    else:
        print(n)
        countdown(n-1)

countdown(3)
\end{lstlisting}
\colb
{\tt
\begin{tabular}{|lrcl|}\hline
\_\_main\_\_ &&& \\\hline\hline
countdown & n & \arrow & 3 \\\hline\hline
countdown & n & \arrow & 2 \\\hline\hline
countdown & n & \arrow & 1 \\\hline\hline
countdown & n & \arrow & 0 \\\hline
\end{tabular}
}

\bigskip

Stack diagram
\colc

\end{frame}

\bfr{Infinite recursion}
\cola
\begin{lstlisting}
def countdown(n):
    print(n)
    countdown(n-1)

countdown(3)
\end{lstlisting}
\colb
{\tt
\begin{tabular}{|lrcl|}\hline
\_\_main\_\_ & &&\\\hline\hline
countdown & n & \arrow & 3 \\\hline\hline
countdown & n & \arrow & 2 \\\hline\hline
countdown & n & \arrow & 1 \\\hline\hline
countdown & n & \arrow & 0 \\\hline\hline
countdown & n & \arrow & -1 \\\hline\hline
countdown & n & \arrow & -2 \\\hline\hline
countdown & n & \arrow & -3 \\\hline\hline
countdown & n & \arrow & -4 \\\hline
\end{tabular}
}
\smallskip

$\ldots$

\bigskip

Stack diagram
\colc
\pause
\bi
\li Must have a base case
\li All other branches must get closer to base case
\ei

\end{frame}

\bfr{Input}
\begin{lstlisting}
>>> text = input()
What are you waiting for?
>>> text
'What are you waiting for?'
\end{lstlisting}

\end{frame}

\bfr{Input, provide a prompt}
\begin{lstlisting}
>>> name = input('What...is your name?\n')
What...is your name?
Arthur, King of the Britons!
>>> name
'Arthur, King of the Britons!'

\end{lstlisting}

\end{frame}

\bfr{Vocabulary}
\begin{description}
\item[floor division:]
An operator, denoted \verb|//|, that divides two numbers and rounds down (toward negative infinity) to an integer.
\item[modulus operator:]
An operator, denoted with a percent sign (\verb|%|), that works on integers and returns the remainder when one number is divided by another.
\item[boolean expression:]
An expression whose value is either True or False.
\item[relational operator:]
One of the operators that compares its operands: 
\verb|==|, \verb|!=|, \verb|>|, \verb|<|, \verb|>=|, and \verb|<=|.
\item[logical operator:]
One of the operators that combines boolean expressions: 
\verb|and|, \verb|or|, and \verb|not|.
\end{description}
\end{frame}
\bfr{Vocabulary}
\begin{description}
\item[conditional statement:]
A statement that controls the flow of execution depending on some condition.
\item[condition:]
The boolean expression in a conditional statement that determines which branch runs.
\item[compound statement:]
A statement that consists of a header and a body. The header ends with a colon (:). The body is indented relative to the header.
\item[branch:]
One of the alternative sequences of statements in a conditional statement.
\item[chained conditional:]
A conditional statement with a series of alternative branches.
\item[nested conditional:]
A conditional statement that appears in one of the branches of another conditional statement.
\end{description}
\end{frame}
\bfr{Vocabulary}
\begin{description}
\item[return statement:]
A statement that causes a function to end immediately and return to the caller.
\item[recursion:]
The process of calling the function that is currently executing.
\item[base case:]
A conditional branch in a recursive function that does not make a recursive call.
\item[infinite recursion:]
A recursion that doesn’t have a base case, or never reaches it. Eventually, an infinite recursion causes a runtime error.
\end{description}
\end{frame}

\end{document}
