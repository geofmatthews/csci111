\documentclass[11pt]{article}
\usepackage{tikz}
\usepackage[margin=1in]{geometry}
\usepackage{amsmath}
\title{Lerping}
\author{Geoffrey Matthews}
\begin{document}
\pagenumbering{gobble}

\maketitle

\begin{center}
\begin{tikzpicture}
%axes
\draw[arrows=<->,thick] (8,0) -- (0,0) -- (0, 6);
%lines
\draw[very thick] (2,2) -- (7,5);
\filldraw (2,2) circle (2pt);
\filldraw (7,5) circle (2pt);
\draw[dashed] (2,0) -- (2,2) -- (0,2);
\draw[dashed] (7,0) -- (7,5) -- (0,5);
\draw (2,0) node[anchor=north] {$a$};
\draw (7,0) node[anchor=north] {$b$};
\draw (0,2) node[anchor=east] {$c$};
\draw (0,5) node[anchor=east] {$d$};

\draw[dashed,color=red] (3,0) -- (3,2.6) -- (0,2.6);
\draw (3,0) node[anchor=north] {$x$};
\draw (0,2.6) node[anchor=east] {$f(x)$};
\end{tikzpicture}
\end{center}

Given a point $x$ between $a$ and $b$,
how can we find the corresponding point between $c$ and $d$?
This problem is called {\em linear interpolation}, or {\bf lerp}
for short.  The basic problem is shown in the figure
and it is easy to see why it's called {\em linear} interpolation.
One approach is to use the two point formula for a line.
The line goes from $(a,c)$ to $(b,d)$, so we have
\begin{align*}
f(x) &= \frac{d-c}{b-a}(x-a) + c
\end{align*}

If you don't remember the two point formula for a line,
it's easy to understand what's going on.  The distance between
$a$ and $b$ is $b-a$.  The distance between $a$ and $x$
is $x-a$.  Therefore, the {\em fraction} of the total distance
is just
\[
\frac{x-a}{b-a}
\]
Now, we want to cover the {\em same fraction} of 
the distance between $c$ and $d$.  The distance
between $c$ and $d$ is just $d-c$.  The fraction of
this distance is, therefore,
\[
\frac{x-a}{b-a}(d-c)
\]
If we take this fraction of the distance between
$c$ and $d$ and add it to $c$, we get the
point we want:
\[
\frac{x-a}{b-a}(d-c) + c
\]
rearranging shows this to be the same number we got from
the two point formula.




\end{document} 