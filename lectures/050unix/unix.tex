\documentclass[12pt]{article}
\usepackage[margin=1in]{geometry}
\usepackage{fancyvrb}
\usepackage{hyperref}
\setlength{\parindent}{0em}
\begin{document}
\section{Python from the command line}

If your python code is in the file {\tt main.py} you can run it from the
command line with:
\begin{Verbatim}[frame=single]
$ python main.py
\end{Verbatim}

For the following python code:
\begin{Verbatim}[frame=single,label=main.py]
import sys
print(sys.argv)
\end{Verbatim}
Running on the command line will produce the following:
\begin{Verbatim}[frame=single]
$ python main.py 1 2 3
['main.py', '1', '2', '3']
\end{Verbatim}



\section{Bash shell commands}

The online tutorial we used is here:
\url{http://www.ee.surrey.ac.uk/Teaching/Unix/}

Below is a partial cheat sheet from

\url{https://github.com/RehanSaeed/Bash-Cheat-Sheet/blob/main/README.md}

You should be familiar with the following unix/bash commands that we covered in class:

\begin{verbatim}

## Navigating Directories

pwd                       # Print current directory path
ls                        # List directories
ls -a|--all               # List directories including hidden
ls -l                     # List directories in long form
ls -l -h|--human-readable # List directories in long form with human readable sizes
ls -t                     # List directories by modification time, newest first
cd foo                    # Go to foo sub-directory
cd                        # Go to home directory
cd ~                      # Go to home directory
cd -                      # Go to last directory

## Directories

mkdir foo                        # Create a directory
mv foo bar                       # Move directory
rmdir foo                        # Delete non-empty directory

## Standard Output, Standard Error and Standard Input

echo "foo" > bar.txt       # Overwrite file with content
echo "foo" >> bar.txt      # Append to file with content

## Moving Files

cp foo.txt bar.txt                                # Copy file
mv foo.txt bar.txt                                # Move file

## Deleting Files

rm foo.txt            # Delete file

## Reading Files

cat foo.txt            # Print all contents
less foo.txt           # Print some contents at a time 
   (g - go to top of file, SHIFT+g, go to bottom of file, /foo to search for 'foo')
head foo.txt           # Print top 10 lines of file
tail foo.txt           # Print bottom 10 lines of file
open foo.txt           # Open file in the default editor
wc foo.txt             # List number of lines words and characters in the file

## File Permissions

| # | Permission              | rwx | Binary |
| - | -                       | -   | -      |
| 7 | read, write and execute | rwx | 111    |
| 6 | read and write          | rw- | 110    |
| 5 | read and execute        | r-x | 101    |
| 4 | read only               | r-- | 100    |
| 3 | write and execute       | -wx | 011    |
| 2 | write only              | -w- | 010    |
| 1 | execute only            | --x | 001    |
| 0 | none                    | --- | 000    |

For a directory, execute means you can enter a directory.

- u - User
- g - Group
- o - Others
- a - All of the above


ls -l foo.sh            # List file permissions
chmod u+x foo.sh         # Give the user execute permission
chmod g+x foo.sh         # Give the group execute permission
chmod u-x,g-x foo.sh     # Take away the user and group execute permission
chmod u+x,g+x,o+x foo.sh # Give everybody execute permission
chmod a+x foo.sh         # Give everybody execute permission
chmod +x foo.sh          # Give everybody execute permission

## Finding Files

Find binary files for a command.

which wget                                 # Find the binary

## Find in Files

grep 'foo' bar.txt                         # Search for 'foo' in file 'bar.txt'

## Disk Usage

df                     # List disks, size, used and available space
du                     # List current directory, subdirectories and file sizes


## Identifying Processes

ps all                 # List all processes
CTRL+Z                 # Suspend a process running in the foreground
bg                     # Resume a suspended process and run in the background
fg                     # Bring the last background process to the foreground
fg 1                   # Bring the background process with the PID to the foreground

sleep 30 &             # Sleep for 30 seconds and move the process into the background
jobs                   # List all background jobs
jobs -p                # List all background jobs with their PID

## Killing Processes

CTRL+C                 # Kill a process running in the foreground
kill PID               # Shut down process by PID gracefully. Sends TERM signal.
kill -9 PID            # Force shut down of process by PID. Sends SIGKILL signal.
\end{verbatim}
\end{document}