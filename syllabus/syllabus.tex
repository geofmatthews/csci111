\documentclass[12pt]{article}
\usepackage[margin=1in]{geometry}
\usepackage{hyperref}
\begin{document}
\centerline{\Large Computer Science 111}
\centerline{\large Fundamentals of Programming 1}

\begin{description}
\item[Instructor:]
~\\ Geoffrey Matthews
\\ Parmly 407A
\\ 540-458-8809
\\ gmatthews@wlu.edu
\item[Webpages:]~
\begin{itemize}
\item For homework and grades: \url{canvas.wlu.edu}
\item For lecture notes and materials: 
\url{https://github.com/geofmatthews/csci111}

If you don't know how to use git, just go to that website,
click the \fbox{\sf Code} button, and then \fbox{\sf Download ZIP}.
This does just what it says.
\item For software: \url{https://www.python.org/}
This software is installed in the labs, but you may want to
download it for your own computer.  It's free.

\end{itemize}
\item[Class lectures:]  MWF 9:45 AM - 10:45 AM, Parmly 405

Lectures are in-person and attendance is required.  If you must miss a
lecture, inform your instructor before the day you must miss
and arrange to get notes from another classmate.  Office hours
will not be used to answer questions, but material already presented
in class will not be reviewed.

\item[Labs:]  R 8:30 AM - 11:30 AM,  Parmly 405

Labs are in-person and attendance is required.
No new material will be presented in labs, but it is a 
unique opportunity to work on the homework with direct 
assistance from the instructor and the TAs.  If you miss
a lab, you will have to make the time up yourself, and it
will likely take you longer.

\item[Office hours:] MWF  11:00 AM - 12:00 PM, Parmly 407A

If you need to see me but
cannot make these hours, please  make an appointment.

\item[Overview:]
This is an introductory course in programming and problem
solving.  Topics include:
\begin{itemize}
\item the design and implementation of algorithms for solving problems
\item an introduction to the syntax, semantics, 
and progmatics of the Python programming language
\item a survey of various types of programming applications such as
numerical computation, text processing, graphics, and image processing
\end{itemize}
 
\item[Objectives:] After taking this course, you should be 
able to
\begin{itemize}
\item apply problem solving skills to a wide variety of computational
problems
\item understand the syntax and semantics of the Python programming language
\item describe a program's functionality in plain English
\item detect, diagnose, and fix errors in a program,
using systematic testing and debugging techniques
\item understand the ethical and historical context of computing
\item undertake further study in computer science
\end{itemize}

\item[Textbook:]
\href{https://greenteapress.com/wp/think-python-2e/}
{\em Think Python 2nd Edition}, by Allen B.~Downey.

This book is available for free online, in both HTML and PDF
formats, from greenteapress.com.  
It is also available in the bookstore and at
amazon.com for those who would prefer a paper copy.

There are many other online resources for studying
Python, feel free to use as many as you think helpful!

\item[Assignments:] Credit will be awarded based on
your performance in the following areas:

\begin{description}
\item[Labs 40\%:] Labs are on Thursdays.  Each lab will be due
the following Tuesday before midnight.  There will be no
late work accepted.  Programs with syntax errors
will receive no credit.  Programs that do not follow
formatting instructions exactly with regard to
naming and commenting will receive no credit.

\item[Quizzes 30\%:] Quizzes will be periodically announced in
class and made available on canvas.  The quizzes are to 
be done without collaboration with your classmates
or any consultation with the internet or other resources.
You may, however, consult your textbook and your own notes.
Each quiz will have a due date posted with the quiz
and will not be accepted after the due date. Quizzes will
usually be discussed in class the day after their due date.

\item[Final exam 20\%:]  The final exam is comprehensive.  It is open
book and open notes, but you may not consult with any classmates
or the internet or other resources.

\item[Class participation 10\%:] During class there will be 
participation quizzes, where we solve problems together
during class time.  These could be individually, in small
groups, or with the entire class.  You are expected to contribute.
I may also take attendance
at any time and award some participation credit for
good attendance. 

\end{description}

\item[Grades:] Grades will be based on the following percentages
out of all possible points for the assignments:

\[
A \geq 90\% > B \geq 80\% > C \geq 70\% > D \geq 60\% > F
\]

The instructor reserves the right to adjust the scale,
but only in a manner that would reward higher grades than
those predicted from the table. Awarding $\pm$ is also at
the discretion of the instructor.


\item[Computer use in class:]
The use of laptops and mobile computing devices are 
permitted during class so long as they are being used 
for the course such as for taking notes and locating
information related to the course. These devices are
not to be used during class for texting, phone calls, 
reading email, social networking, completing assignments
for other courses, shopping, or any other topic unrelated
to the class you are currently attending.

\item[Accommodations]
Washington and Lee University makes reasonable academic 
accommodations for qualified students with disabilities. 
All undergraduate accommodations must be approved through 
the Office of the Dean of the College. Students requesting 
accommodations for this course should present an official 
accommodation letter within the first two weeks of the 
(fall or winter) term and schedule a meeting outside of 
class time to discuss accommodations. It is the student’s 
responsibility to present this paperwork in a timely 
fashion and to follow up about accommodation arrangements. 
Accommodations for test-taking should be arranged with the 
professor at least a week before the date of the test or exam.

     
\item[Academic dishonesty:] Please review the university's
honor system, and the definition of plagiarism
which can be found at

\centerline{
\url{https://my.wlu.edu/executive-committee/the-honor-system}
}

  Unless specified otherwise, all work for this course is meant to
  be done {\bf individually.}  The work that you turn in for a grade
  must be completely your own, or you will be guilty of academic
  dishonesty.

  Nevertheless, it is a valiable learning experience to discuss
  work with your fellow students, and this is encouraged.
  However, after working with a colleague, {\bf you may not keep any
    paper or electronic copies of anything you produced together!}
  You may only keep your memories.  In particular, this means that
  {\bf you may not ask for or give help while sitting in front of a
    computer where the assignment is open!}  Also, {\bf you may not
    use anything a colleague has emailed to you!}  Delete the email
  and do not save a copy.

  To help understand what I mean, remember the \\
  \centerline{\fbox{\sf Long Term
    Memory Rule}}  You may discuss, sketch, write things down, use
  your computers, whatever, but after you are done working with your
  fellow students all files must be deleted, whiteboards erased, and
  all papers you created must be destroyed.  You should then watch a
  rerun of {\em the Simpson's}, play a game of ping-pong, take a walk,
  or something else for half an hour. After this you can go back to
  your assignment (alone) and use the knowledge you have now gained.

  We are here to help you get a great education.  Please do not
  put us in a situation where we have to police you for plagiarism.
  We hate that.


\end{description}

\end{document}