\documentclass[12pt]{article}
\usepackage[margin=1in]{geometry}
\usepackage{fancyvrb}
\usepackage{multicol}
\usepackage{hyperref}
\usepackage{amsmath}
\usepackage{amsfonts}

\usepackage[listings]{tcolorbox}

\definecolor{codegreen}{rgb}{0,0.6,0}
\definecolor{codegray}{rgb}{0.5,0.5,0.5}
\definecolor{codepurple}{rgb}{0.58,0,0.82}
\definecolor{backcolour}{rgb}{0.95,0.95,0.92}

\lstdefinestyle{mystyle}{
    language=Python,
    backgroundcolor=\color{backcolour},   
    commentstyle=\color{codegreen},
    keywordstyle=\color{magenta},
    numberstyle=\tiny\color{codegray},
    stringstyle=\color{codepurple},
    basicstyle=\ttfamily\footnotesize,
    breakatwhitespace=false,         
    breaklines=true,                 
    captionpos=b,                    
    keepspaces=true,                 
    numbers=left,                    
    numbersep=5pt,                  
    showspaces=false,                
    showstringspaces=false,
    showtabs=false,                  
    tabsize=2,
    escapechar=|,
    frame=single
}

\lstset{style=mystyle}

\newcommand{\showfig}[2]{
\noindent\includegraphics[width=\textwidth]{#1}
\centerline{#1}
}

\begin{document}
\sloppy
\centerline{\Large CSCI 111, Bonus Lab 1}
\centerline{\large Add Variables to the RPN Calculator}


\begin{description}

\item[Individual work:]  All work must be your own.  Do not share
code with anyone other than the instructor and teaching assistants.
This includes looking over shoulders at screens with the code open.
You may discuss ideas, algorithms, approaches, {\em etc.} with
other students but NEVER actual code.

\item[RPN with variables:]
Add the ability to define variables
to the RPN calculator.  You will need to have a working
RPN calculator for this!

Variables are defined with the \lstinline{def}
operator.  For example, in this session we
find the sine of 3.14159, and then we define
the variable \lstinline{pi} to have the value
3.14159 and take the sine of \lstinline{pi}
to get the same value:
\begin{lstlisting}
RPN>>> 3.14159 sin
Stack: [2.65358979335273e-06] Tokens: [3.14159, 'sin']
Stack: [2.65358979335273e-06, 3.14159] Tokens: ['sin']
[2.65358979335273e-06, 2.65358979335273e-06]
RPN>>> clr
Stack: [2.65358979335273e-06, 2.65358979335273e-06] Tokens: ['clr']
[]
RPN>>> 3.14159 pi def
Stack: [] Tokens: [3.14159, 'pi', 'def']
Stack: [3.14159] Tokens: ['pi', 'def']
Stack: [3.14159, 'pi'] Tokens: ['def']
[]
RPN>>> pi sin
Stack: [] Tokens: ['pi', 'sin']
Stack: ['pi'] Tokens: ['sin']
[2.65358979335273e-06]
RPN>>> 
\end{lstlisting}
\item[The store:]
Variables and their values are kept in a global
variable called \lstinline{store}.\footnote{Think ``storehouse''
and not ``marketplace.''}  The store is a dictionary
mapping variables to values.  

\item[Parsing:] You will have to account for
variables in your tokenizer.  Variables
are not keywords, but consist of all 
alphabetic letters, and are represented
after tokenizing just by the string, like keywords.

\item[Processing:]
When interpreting the tokenized list 
variables are passed onto the stack just like
numbers.

When the new keyword \lstinline{def} is encountered,
two items are popped from the stack, the variable
should be on top, and its value underneath.  The
\lstinline{store} dictionary stores the value
under the variable as key.

When an item is removed from the stack,
its value is found.  The value of a number
is the number, the value of a string (variable)
is the value found in the \lstinline{store}.

\item[Errors:] You do not have to handle
any errors, such as a variable not defined, etc.


\end{description}

\end{document}
