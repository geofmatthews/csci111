\documentclass[12pt]{article}
\usepackage[margin=1in]{geometry}
\usepackage{fancyvrb}
\usepackage{multicol}
\usepackage{hyperref}
\usepackage{amsmath}
\usepackage{amsfonts}

\usepackage[listings]{tcolorbox}

\definecolor{codegreen}{rgb}{0,0.6,0}
\definecolor{codegray}{rgb}{0.5,0.5,0.5}
\definecolor{codepurple}{rgb}{0.58,0,0.82}
\definecolor{backcolour}{rgb}{0.95,0.95,0.92}

\lstdefinestyle{mystyle}{
    language=Python,
    backgroundcolor=\color{backcolour},   
    commentstyle=\color{codegreen},
    keywordstyle=\color{magenta},
    numberstyle=\tiny\color{codegray},
    stringstyle=\color{codepurple},
    basicstyle=\ttfamily\footnotesize,
    breakatwhitespace=false,         
    breaklines=true,                 
    captionpos=b,                    
    keepspaces=true,                 
    numbers=left,                    
    numbersep=5pt,                  
    showspaces=false,                
    showstringspaces=false,
    showtabs=false,                  
    tabsize=2,
    escapechar=|,
    frame=single
}

\lstset{style=mystyle}

\newcommand{\showfig}[2]{
\noindent\includegraphics[width=\textwidth]{#1}
\centerline{#1}
}

\begin{document}
\sloppy
\centerline{\Large CSCI 111, Bonus Lab 2}
\centerline{\large Add list processing to the RPN Calculator}


\begin{description}

\item[Individual work:]  All work must be your own.  Do not share
code with anyone other than the instructor and teaching assistants.
This includes looking over shoulders at screens with the code open.
You may discuss ideas, algorithms, approaches, {\em etc.} with
other students but NEVER actual code.

\item[RPN with lists:]

Add lists to the RPN calculator.  You will need a working
RPN calculator for this!

\item[List creator keywords:]
You will need to add two keywords to the interpreter:
\begin{itemize}
\item \lstinline{[}
\item \lstinline{]}
\end{itemize}

These will be tokenized like regular keywords.

\item[Processing list creator keywords:]
When processing items from the tokenized list,
the \lstinline{[} operator is simply placed
on top of the stack.

When the \lstinline{]} operator is found,
items are taken one by one off the stack
and placed in a new list.  When the
\lstinline{[} operator is found on the
stack, we stop adding items to the list
and place the new list on the stack.
We can see this in operation here:
\begin{lstlisting}
RPN>>> [ 1 2 3 ]
Stack: [] Tokens: ['[', 1, 2, 3, ']']
Stack: ['['] Tokens: [1, 2, 3, ']']
Stack: ['[', 1] Tokens: [2, 3, ']']
Stack: ['[', 1, 2] Tokens: [3, ']']
Stack: ['[', 1, 2, 3] Tokens: [']']
[[1, 2, 3]]
RPN>>> clr
Stack: [[1, 2, 3]] Tokens: ['clr']
[]
RPN>>> 99 [ 3 9 12 ] 22 [ 4 3 1 ]
Stack: [] Tokens: [99, '[', 3, 9, 12, ']', 22, '[', 4, 3, 1, ']']
Stack: [99] Tokens: ['[', 3, 9, 12, ']', 22, '[', 4, 3, 1, ']']
Stack: [99, '['] Tokens: [3, 9, 12, ']', 22, '[', 4, 3, 1, ']']
Stack: [99, '[', 3] Tokens: [9, 12, ']', 22, '[', 4, 3, 1, ']']
Stack: [99, '[', 3, 9] Tokens: [12, ']', 22, '[', 4, 3, 1, ']']
Stack: [99, '[', 3, 9, 12] Tokens: [']', 22, '[', 4, 3, 1, ']']
Stack: [99, [3, 9, 12]] Tokens: [22, '[', 4, 3, 1, ']']
Stack: [99, [3, 9, 12], 22] Tokens: ['[', 4, 3, 1, ']']
Stack: [99, [3, 9, 12], 22, '['] Tokens: [4, 3, 1, ']']
Stack: [99, [3, 9, 12], 22, '[', 4] Tokens: [3, 1, ']']
Stack: [99, [3, 9, 12], 22, '[', 4, 3] Tokens: [1, ']']
Stack: [99, [3, 9, 12], 22, '[', 4, 3, 1] Tokens: [']']
[99, [3, 9, 12], 22, [4, 3, 1]]
RPN>>> 
\end{lstlisting}

\item[List processing commands:]

Add the following keywords to the RPN calculator:
\begin{itemize}
\item \lstinline{ref}:  Taking a list and
an integer on top of the stack, pushes the
indexed item onto the list.
\item \lstinline{del}
Taking a list and an integer off the stack,
deletes the indexed item and pushes the new
list on the stack.
\item \lstinline{append}
Taking a list and anything else on the list,
pushes the list with the new item onto the stack.
\item \lstinline{len}
Taking a list off the stack, pushes the
list and its length back on the stack.
\end{itemize}

Note that since \lstinline{+} already works 
with lists, we get that for free.

Examples of this at work are seen here:
\begin{lstlisting}
RPN>>> [ 1 2 3 4 ] len 1 - del
Stack: [] Tokens: ['[', 1, 2, 3, 4, ']', 'len', 1, '-', 'del']
Stack: ['['] Tokens: [1, 2, 3, 4, ']', 'len', 1, '-', 'del']
Stack: ['[', 1] Tokens: [2, 3, 4, ']', 'len', 1, '-', 'del']
Stack: ['[', 1, 2] Tokens: [3, 4, ']', 'len', 1, '-', 'del']
Stack: ['[', 1, 2, 3] Tokens: [4, ']', 'len', 1, '-', 'del']
Stack: ['[', 1, 2, 3, 4] Tokens: [']', 'len', 1, '-', 'del']
Stack: [[1, 2, 3, 4]] Tokens: ['len', 1, '-', 'del']
Stack: [[1, 2, 3, 4], 4] Tokens: [1, '-', 'del']
Stack: [[1, 2, 3, 4], 4, 1] Tokens: ['-', 'del']
Stack: [[1, 2, 3, 4], 3] Tokens: ['del']
[[1, 2, 3]]
RPN>>> [ 4 5 6 ] +
Stack: [[1, 2, 3]] Tokens: ['[', 4, 5, 6, ']', '+']
Stack: [[1, 2, 3], '['] Tokens: [4, 5, 6, ']', '+']
Stack: [[1, 2, 3], '[', 4] Tokens: [5, 6, ']', '+']
Stack: [[1, 2, 3], '[', 4, 5] Tokens: [6, ']', '+']
Stack: [[1, 2, 3], '[', 4, 5, 6] Tokens: [']', '+']
Stack: [[1, 2, 3], [4, 5, 6]] Tokens: ['+']
[[1, 2, 3, 4, 5, 6]]
RPN>>> 0 del
Stack: [[1, 2, 3, 4, 5, 6]] Tokens: [0, 'del']
Stack: [[1, 2, 3, 4, 5, 6], 0] Tokens: ['del']
[[2, 3, 4, 5, 6]]
RPN>>> 
\end{lstlisting}

\end{description}

\end{document}
